\clearpage

\section{Rezende et al.: Normalizing flows for spheres and tori}

\begin{notebox}
\textbf{Paper: } \fullcite{rezendeNormalizingFlowsTori2020}

\hfill Notes taken: 12/2/2020 \index{February 2020}
\end{notebox}

\begin{notebox}
\tldr Normalizing flows over non-Euclidean data should take into account the particular topology of the space and use specific diffeomorphic transformations. If you project to $\mR^D$ to do the flows there, the projection itself may not be diffeomorphic and therefore the whole chain of transformations in the change of variable formula breaks. Quite a lot of math bruhaha for modelling data on spheres and tori.
\end{notebox}

\subsection{Intro}

Use normalizing flows, that is the usual change of variable rule for density, e.g. equation \eqref{eqpre:changeOfVarMulti} over data in some non-Euclidean space $\mM$.
The simple solution of applying the flow in $\mR^D$ and then project to $\mM$ is problematic when $\mM$ is not diffeomorphic to $\mR^D$.

The simple solution of applying the flow in $\mR^D$ and then project to $\mM$ is problematic if $\mM$ and $\mR^D$ are not diffeomorphic.

They focus on data on a sphere $\mS^D$\index{sphere} or tori\index{tori} $\mT^D$. A circle $\mS^1$ in $\mR^2$ can be parametrised either by $\{(x_1, x_2) \in \mR^2 : x_1^2 + x_2^2 = 1\}$ or by $\theta \in [0, 2\pi]$.
They give conditions for a valid diffeomorphism $f : [0, 2\pi] \to [0, 2\pi]$ taking into account the periodic nature of the circle.
They then discuss three types of diffeomorphisms on circle: M\:{o}bius transformations\index{M\:{o}bius transformations}, circular splines\index{circular splines} and non-compact projections\index{non-compact projections}.

They explain how to combine these either by composition or as convex combinations to increase the expressivity of the transformed distributions and the complexity of evaluating the final transform $f$ and its inverse $f^{-1}$. Sometimes these cannot be done analytically and require numerical methods.

Generalizations to torus are based on autoregressive flows over the circle transformers.
To extend to higher-dimensional spheres they use recursive construction based on exponential map and cylindrical coordinates.

It is all very mathematical (space topology) and I'm not sure how useful. It seems they want to extend this to Li groups that or of interest in fundamental physics (particle interactions).




