\section{Change of variables at density level}
\begin{frame}{From push-forward to change of variables}

\structure{What we saw:} uniform distribution + affine transformation
\begin{itemize}
\item $X \sim \text{Uniform}[0,1]$, $g(x) = 2x + 3$, $Y = g(X)$
\item stretching by factor 2 $\Rightarrow$ density halves
\item easy to compute: $p_Y = \frac{1}{2}$
\end{itemize}

\vskip 1em
\structure{General question:}\\
Given arbitrary distribution $p_X$ and transformation $g$, how do we compute $p_Y$?

\vskip 1em
\structure{Challenge:}
\begin{itemize}
\item not just uniform distributions
\item not just affine transformations
\item need general formula relating $p_Y$ to $p_X$ and $g$
\end{itemize}

\vskip 0.5em
\alert{Goal: derive the change of variables formula with Jacobian determinant}

\end{frame}

\note[enumerate]
{
\item The uniform example was special - constant density, linear transformation
\item In ML: we use complex distributions (Gaussian, data distributions) and complex transformations (neural networks)
\item Need to understand how densities transform in general
\item This is the change of variables formula - fundamental for normalizing flows, diffusion models, etc.
}

\begin{frame}{Step 1: Start from push-forward}

\structure{Recall:} push-forward at measure level
\[
P_Y(C) = P_X(g^{-1}(C))
\]

\vskip 1em
\structure{Express using densities:} (assuming densities exist)
\[
P_Y(C) = \int_C p_Y(y) \, dy = \int_{g^{-1}(C)} p_X(x) \, dx = P_X(g^{-1}(C))
\]

\vskip 1em
\structure{Key observation:}
\begin{itemize}
\item left side: integrate $p_Y$ over $C$ in $y$-space
\item right side: integrate $p_X$ over $g^{-1}(C)$ in $x$-space
\item same probability, different spaces
\end{itemize}

\vskip 0.5em
\structure{Idea:} change variables in right side from $x$ to $y = g(x)$

\end{frame}

\note[enumerate]
{
\item This is the starting point - push-forward expressed with densities
\item Left side: probability in target space $U$ (using $p_Y$)
\item Right side: probability in source space $T$ (using $p_X$)
\item They're equal by push-forward definition
\item Next: use change of variables from calculus to transform the right integral
}

\begin{frame}{Step 2: Change of variables in integrals}

\structure{From calculus:} change of variables formula
\[
\int_{g^{-1}(C)} f(x) \, dx = \int_C f(g^{-1}(y)) \left| \det \frac{\partial g^{-1}}{\partial y}(y) \right| \, dy
\]

\vskip 0.5em
Assuming $g: \mR^d \to \mR^d$ is invertible and differentiable

\vskip 1em
\structure{The Jacobian determinant:}
\begin{itemize}
\item $\frac{\partial g^{-1}}{\partial y}$ is the $d \times d$ Jacobian matrix of $g^{-1}$
\item $\det \frac{\partial g^{-1}}{\partial y}$ measures local volume change
\item absolute value: $\left| \det \frac{\partial g^{-1}}{\partial y} \right|$
\end{itemize}

\vskip 1em
\structure{Intuition:} $dx = \left| \det \frac{\partial g^{-1}}{\partial y} \right| dy$ (infinitesimal volume scaling)

\end{frame}

\note[enumerate]
{
\item This is the standard change of variables from multivariable calculus
\item The Jacobian matrix contains all partial derivatives
\item Its determinant tells us how volumes scale under the transformation
\item We need absolute value because we're integrating (measure must be positive)
\item For $d=1$: this reduces to $|dx/dy|$ or $|1/g'(x)|$
\item We can also write in terms of $g$ instead of $g^{-1}$ (next slide)
}

\begin{frame}{Step 3: Apply to our setting}

\structure{Apply change of variables:}
\[
\int_{g^{-1}(C)} p_X(x) \, dx = \int_C p_X(g^{-1}(y)) \left| \det \frac{\partial g^{-1}}{\partial y}(y) \right| \, dy
\]

\vskip 1em
\structure{But we also have:}
\[
\int_{g^{-1}(C)} p_X(x) \, dx = \int_C p_Y(y) \, dy
\]
by push-forward

\vskip 1em
\structure{Therefore:}
\[
\int_C p_Y(y) \, dy = \int_C p_X(g^{-1}(y)) \left| \det \frac{\partial g^{-1}}{\partial y}(y) \right| \, dy
\]

\vskip 0.5em
Since this holds for all $C$, the integrands must be equal!

\end{frame}

\note[enumerate]
{
\item This is the key step: combining push-forward with change of variables
\item We have two expressions for the same integral
\item Both integrate over $C$ in $y$-space
\item The integrands must be equal
\item This gives us the formula for $p_Y$ in terms of $p_X$
}

\begin{frame}{Step 4: Change of variables formula}

\structure{Result:}
\[
p_Y(y) = p_X(g^{-1}(y)) \left| \det \frac{\partial g^{-1}}{\partial y}(y) \right|
\]

\vskip 1em
\structure{Alternative form:} using inverse function theorem
\[
\det \frac{\partial g^{-1}}{\partial y}(y) = \frac{1}{\det \frac{\partial g}{\partial x}(g^{-1}(y))}
\]

Therefore:
\[
p_Y(y) = p_X(g^{-1}(y)) \left| \det \frac{\partial g}{\partial x}(g^{-1}(y)) \right|^{-1}
\]

\vskip 1em
\structure{Common notation:} $J_g(x) = \frac{\partial g}{\partial x}(x)$ (Jacobian of $g$)
\[
\boxed{p_Y(y) = \frac{p_X(g^{-1}(y))}{|\det J_g(g^{-1}(y))|}}
\]

\end{frame}

\note[enumerate]
{
\item This is the change of variables formula!
\item First form: uses Jacobian of $g^{-1}$ (inverse transformation)
\item Second form: uses Jacobian of $g$ (forward transformation) - usually more convenient
\item Inverse function theorem: $J_{g^{-1}}(y) = [J_g(g^{-1}(y))]^{-1}$
\item Determinants: $\det(A^{-1}) = 1/\det(A)$
\item The Jacobian determinant in denominator - this is key for normalizing flows
}

\begin{frame}{Geometric interpretation of Jacobian}

\structure{What does $|\det J_g(x)|$ mean geometrically?}

\vskip 0.5em
Local volume scaling factor at point $x$

\vskip 1em
\structure{In 1D:} $g: \mR \to \mR$
\[
|\det J_g(x)| = |g'(x)| = \text{local stretching factor}
\]

\vskip 1em
\structure{Example:} $g(x) = 2x + 3$
\begin{itemize}
\item $g'(x) = 2$ everywhere
\item stretches by factor 2 everywhere
\item density: $p_Y(y) = \frac{p_X(g^{-1}(y))}{2}$ (matches our uniform example!)
\end{itemize}

\vskip 1em
\structure{In higher dimensions:}
\begin{itemize}
\item $|\det J_g(x)|$ measures how $g$ stretches/compresses volume near $x$
\item $|\det J_g(x)| > 1$: expansion $\Rightarrow$ density decreases
\item $|\det J_g(x)| < 1$: contraction $\Rightarrow$ density increases
\end{itemize}

\end{frame}

\note[enumerate]
{
\item The Jacobian determinant has clear geometric meaning
\item It tells us how volumes change under the transformation
\item In 1D: just the derivative (slope)
\item Our uniform example: constant Jacobian = 2, density halves
\item In general: Jacobian varies with position $x$
\item This is why neural networks can create complex distributions - varying Jacobian
\item Key principle: stretching space → lower density, compressing space → higher density
}