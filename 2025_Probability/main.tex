\documentclass[aspectratio=169, smaller]{beamer}
\setbeameroption{hide notes}
\mode<presentation>
\usepackage[]{slidespreamble}
\usepackage[]{math_commands}
\usepackage{layout}

% presentation title
\title{Probability basics}

\begin{document}

\begin{frame}[t,plain]

%%%% Logos %%%%
\begin{textblock}{2}[0,0](0.5,0.5)
\includegraphics[width=1\textwidth]{id23_cairo-thws_logo_100pt_en_orange_NEU.png}
\end{textblock}

\begin{textblock}{10}[0,0](0,1.5)
\centering
{\Large Magda Gregorová}\\
\href{mailto:magda.gregorova@thws.de}{magda.gregorova@thws.de}

\vskip 2em
{\LARGE \color{thwspetrol} \inserttitle}\\
{\large \color{thwspetrol} \insertsubtitle}

\vskip 1em
\today

\vskip 2em
Licensed according to \href{https://creativecommons.org/licenses/by-sa/4.0}{CC-BY-SA 4.0}
\end{textblock}

\end{frame}



\begin{frame}
\layout
\end{frame}

\begin{frame}[t]{Outline}
\setcounter{framenumber}{1}
Blabla
\end{frame}

\begin{frame}{Probability space}

\structure{Probability space $(S, \salg, \mP)$:}
\begin{itemize}
  \item measurable space $(S, \salg)$
  \begin{itemize}
    \item $S$ - sample space
    \item $\salg$ - $\sigma$-algebra on $S$ - collection of subsets
  \end{itemize}
  \item probability measure $\mP$ - real-valued function on sample space $(S, \salg)$ s.t.:
  \begin{itemize}
    \item non-negativity: $\mP(A) \geq 0$ for all $A \in \salg$
    \item countable additivity: countable disjoint $\{A_i: i \in I\} \in \salg \Rightarrow \mP\left( \bigcup_{i \in I} A_i \right) = \sum_{i \in I} \mP(A_i)$
    \item \alert{normalization: $\mP(S) = 1$}
  \end{itemize}
\end{itemize}

\vskip 1em
\structure{Note:}
any finite positive measure $\mu$ on $(S, \mathscr{S})$ $\Rightarrow$ prob. measure $\mP(A) = \mu(A) / \mu(S)$.


\end{frame}

\note[enumerate]
{
\item $\sigma$-algebra $\salg$ - non-empty collection of subsets closed under complement and countable unions 
\vskip -0.1em
\begin{itemize}
    \item if $A \in \salg$, then $A^c \in \salg$
    \item if $A_i \in \salg$ for $i \in I$ (countable index set), then $\bigcup_{i \in I} A_i \in \salg$
    \item if $A \in \salg$ and $A \in \salg$, then $A \cup B \in \salg$
    \item in consequence $\emptyset \in \salg$ and $S \in \salg$
  \end{itemize}
\item $(S, \salg)$ forms a measurable space in this context called the sample space.
\item Probability measure is the same as probability distribution or probability law
\item More generally a positive measure on $(S, \salg)$ is a function $\mu: \salg \to [0, \infty]$ satisfying non-negativity and countable additivity. A probability measure is a positive measure with total measure equal to 1. 
\item The triplet $(S, \salg, \mu)$ is a measure space. Probability space is a special case of a measure space where the total measure is 1.  
\item Any finite positive measure $\mu$ on the sample space $(S, \mathscr{S})$ can be re-scaled into a probability measure as $\mP(A) = \mu(A) / \mu(S), \ A \in \mathscr{S}$ $\Rightarrow$ link to energy models.
}

\begin{frame}{Positive measure}

\structure{Positive measure on $(S, \salg)$ - function $\mu: \salg \to [0, \infty]$ s.t.:}
\begin{itemize}
  \item $\mu(\emptyset) = 0$
  \item countable additivity: countable disjoint $\{A_i: i \in I\} \in \salg \Rightarrow \mu\left( \bigcup_{i \in I} A_i \right) = \sum_{i \in I} \mu(A_i) $
  \item $\Rightarrow$ measure space $(S, \salg, \mu)$
\end{itemize}

\vskip 1em
\structure{Note:}
if $\mu(S) < \infty \Rightarrow ((S, \salg, \mu))$ \alert{finite} measure space.\\
\hspace*{6ex} if $\mu(S) = 1 \Rightarrow ((S, \salg, \mu))$ \alert{probability} (measure) space.

\begin{block}{Important:}
adslfkj\\
ad;lfkj
\end{block}

 
\end{frame}

\begin{frame}{Random variables}

\structure{Random variable $X: S \to T$ - measurable function $S$ to $T$}
\begin{itemize}
\item $(S, \salg, \mP)$ - probability space
\item $(T, \mathscr{T})$ - another measurable space
\item for outcome $s \in S$, $X$ takes value $x = X(s) \in T$ - realization of r.v. $X$
\item pre-image of $x \in T$: $X^{-1}(x) = \{s \in S: X(s) = x\} \in \salg$
\item pre-image of $B \in T$: $X^{-1}(B) = \{s \in S: X(s) \in B\} \in \salg$
\end{itemize}


  
\end{frame}

% \begin{frame}{Definition of a Probability Space}
    A \textbf{probability space} is a mathematical framework for modeling random experiments
    and consists of three components:

    \begin{itemize}
        \item \textbf{Sample Space} (\(\Omega\)): 
        \begin{itemize}
            \item The set of all possible outcomes of a random experiment.
        \end{itemize}
        
        \item \textbf{Sigma-Algebra} (\(\mathcal{F}\)): 
        \begin{itemize}
            \item A collection of subsets of \(\Omega\), called events, that includes \(\Omega\) itself, is closed under complementation and countable unions.
        \end{itemize}
        
        \item \textbf{Probability Measure} (\(P\)): 
        \begin{itemize}
            \item A function \(P: \mathcal{F} \to [0, 1]\) that assigns a probability to each event in \(\mathcal{F}\).
            \item Satisfies: \(P(\Omega) = 1\) and is \(\sigma\)-additive.
        \end{itemize}
    \end{itemize}
    
    \textbf{Notation:} The probability space is denoted by \((\Omega, \mathcal{F}, P)\).
\end{frame}


\end{document}