\section{Integration with respect to measures}

\begin{frame}{Lebesgue integration - the idea}

\structure{Riemann integral} - partition the \alert{domain} (x-axis)
\begin{itemize}
  \item approximate area by rectangles with base on x-axis
  \item $\int_a^b f(x) \, dx \approx \sum f(x_i) \Delta x_i$
\end{itemize}

\vskip 1em
\structure{Lebesgue integral} - partition the \alert{range} (y-axis)
\begin{itemize}
  \item approximate by measuring "how much" of domain maps to each range value
  \item $\int_S f \, d\mu \approx \sum y_i \cdot \mu(\{s: f(s) \approx y_i\})$
\end{itemize}

\vskip 1em
\structure{Key point:} integrate with respect to a measure $\mu$, not just "dx"

\end{frame}

\note[enumerate]{
\item Don't need to go into technical details of construction
\item Main idea: measure "slices" of the domain by the values the function takes
\item Works for much more general functions than Riemann integral
\item The measure $\mu$ tells us "how much weight" each set has
}

\begin{frame}{Integration notation}

\structure{Measure space $(S, \salg, \mu)$, measurable function $f: S \to \mathbb{R}$}

\end{frame}

\note[enumerate]{
\item Multiple notations exist - they all mean the same thing
\item The $\mu(ds)$ notation emphasizes we're integrating with respect to measure $\mu$
\item Integrability means the integral is finite
}

