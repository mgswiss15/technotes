\section{Integration with respect to measures}

\begin{frame}{Measurable functions}

\structure{Measure space $(S, \salg, \mu)$, measurable space $(T, \mathscr{T})$}

\vskip 0.5em
\structure{Measurable function $f: S \to T$}
\begin{itemize}
  \item for any $B \in \mathscr{T}$, the pre-image $f^{-1}(B) \in \salg$
  \item continuous functions are measurable
\end{itemize}

\vskip 1em
\structure{Real-valued measurable functions $f: S \to \mathbb{R}$}
\begin{itemize}
  \item $\{f \leq a\} = \{s \in S: f(s) \leq a\} \in \salg$ for all $a \in \mathbb{R}$
  \item equivalently: $\{f < a\}$, $\{f \geq a\}$, $\{f > a\}$, $\{f = a\}$ all in $\salg$
  \item operations: if $f, g$ measurable, then $f + g$, $fg$, $\max(f,g)$, $\min(f,g)$ measurable
  \item limits: if $f_n$ measurable for all $n$, then $\sup_n f_n$, $\inf_n f_n$, $\limsup_n f_n$, $\liminf_n f_n$ measurable
\end{itemize}

\end{frame}

\note[enumerate]{
\item A function $f: S \to \mathbb{R}$ is measurable if the pre-image of any Borel set is in $\salg$.
\item For real-valued functions, it suffices to check that $\{f \leq a\} \in \salg$ for all $a \in \mathbb{R}$.
\item The class of measurable functions is closed under pointwise limits, which is crucial for integration theory.
\item Measurability is preserved under algebraic operations and pointwise limits.
}

\begin{frame}{Simple functions}

\structure{Simple function - measurable function taking finitely many values}

\vskip 0.5em
For measure space $(S, \salg, \mu)$, simple function $\phi: S \to \mathbb{R}$:
$$\phi(s) = \sum_{i=1}^{n} a_i \mathbf{1}_{A_i}(s), \quad a_i \in \mathbb{R}, \, A_i \in \salg$$

where $\mathbf{1}_{A}(s) = \begin{cases} 1 & \text{if } s \in A \\ 0 & \text{if } s \notin A \end{cases}$ is the \alert{indicator function}

\vskip 1em
\structure{Properties:}
\begin{itemize}
  \item $A_i$ are disjoint: $A_i \cap A_j = \emptyset$ for $i \neq j$
  \item $\bigcup_{i=1}^{n} A_i = S$ (can always be arranged by adding zero terms)
  \item any non-negative measurable function can be approximated by simple functions
\end{itemize}

\vskip 1em
\structure{Standard form:} $\phi = \sum_{i=1}^{n} a_i \mathbf{1}_{A_i}$ with distinct $a_i$ and disjoint $A_i$

\end{frame}

\note[enumerate]{
\item Simple functions are the building blocks of integration theory.
\item The indicator function $\mathbf{1}_A$ is also called the characteristic function of $A$.
\item Any simple function can be written in standard form where all $a_i$ are distinct and the sets $A_i$ partition $S$.
\item The approximation of measurable functions by simple functions is fundamental - it allows us to define integration for general measurable functions.
}

\begin{frame}{Integration of simple functions}

\structure{Measure space $(S, \salg, \mu)$, simple function $\phi = \sum_{i=1}^{n} a_i \mathbf{1}_{A_i}$}

\vskip 1em
\structure{Integral of $\phi$ with respect to $\mu$:}
$$\int_S \phi \, d\mu = \sum_{i=1}^{n} a_i \mu(A_i)$$

\vskip 1em
\structure{Properties:}
\begin{itemize}
  \item \alert{linearity}: for $\alpha, \beta \in \mathbb{R}$, $\int (\alpha \phi + \beta \psi) d\mu = \alpha \int \phi d\mu + \beta \int \psi d\mu$
  \item \alert{monotonicity}: if $\phi \leq \psi$, then $\int \phi d\mu \leq \int \psi d\mu$
  \item \alert{positivity}: if $\phi \geq 0$, then $\int \phi d\mu \geq 0$
  \item for $A \in \salg$: $\int_A \phi d\mu = \int_S \phi \mathbf{1}_A d\mu = \sum_{i=1}^{n} a_i \mu(A_i \cap A)$
\end{itemize}

\vskip 0.5em
\structure{Note:} definition well-defined (independent of representation of $\phi$)

\end{frame}

\note[enumerate]{
\item The integral of a simple function is defined as a weighted sum of the measures of the sets where the function takes constant values.
\item The definition is well-defined because if we write the same simple function in different ways, we get the same integral value.
\item These properties (linearity, monotonicity, positivity) are fundamental and will extend to general measurable functions.
\item The notation $\int_A \phi d\mu$ means we integrate $\phi$ over the subset $A$.
}

\begin{frame}{Integration of non-negative measurable functions}

\structure{Measure space $(S, \salg, \mu)$, non-negative measurable $f: S \to [0, \infty]$}

\vskip 1em
\structure{Lebesgue integral of $f$:}
$$\int_S f \, d\mu = \sup \left\{ \int_S \phi \, d\mu : \phi \text{ simple}, \, 0 \leq \phi \leq f \right\}$$

\vskip 0.5em
approximation from below by simple functions

\vskip 1em
\structure{Properties (inherited from simple functions):}
\begin{itemize}
  \item \alert{linearity}: for $\alpha, \beta \geq 0$, $\int (\alpha f + \beta g) d\mu = \alpha \int f d\mu + \beta \int g d\mu$
  \item \alert{monotonicity}: if $f \leq g$, then $\int f d\mu \leq \int g d\mu$
  \item \alert{positivity}: $\int f d\mu \geq 0$
  \item if $f = 0$ a.e., then $\int f d\mu = 0$
\end{itemize}

\vskip 0.5em
\structure{Note:} $\int f d\mu$ may be $+\infty$

\end{frame}

\note[enumerate]{
\item The Lebesgue integral extends the integral from simple functions to all non-negative measurable functions by taking the supremum over all simple function approximations from below.
\item This definition ensures that if $f$ itself is a simple function, we recover the previous definition.
\item The integral can be infinite, which is allowed in measure theory.
\item The almost everywhere property is crucial: functions that differ only on a null set have the same integral.
\item For non-negative functions, linearity holds for non-negative scalars.
}

\begin{frame}{Integration of general measurable functions}

\structure{Measure space $(S, \salg, \mu)$, measurable $f: S \to \mathbb{R}$}

\vskip 1em
\structure{Positive and negative parts:}
$$f^+(s) = \max(f(s), 0), \quad f^-(s) = \max(-f(s), 0)$$

Properties: $f^+, f^- \geq 0$ measurable, $f = f^+ - f^-$, $|f| = f^+ + f^-$

\vskip 1em
\structure{Integrable function - $f$ s.t. $\int |f| d\mu < \infty$}

\vskip 0.5em
\structure{Integral of integrable $f$:}
$$\int_S f \, d\mu = \int_S f^+ \, d\mu - \int_S f^- \, d\mu$$

both integrals finite $\Rightarrow$ difference well-defined

\vskip 1em
\structure{Space of integrable functions:} $L^1(S, \salg, \mu) = \{f: S \to \mathbb{R} \text{ measurable}, \, \int |f| d\mu < \infty\}$

\end{frame}

\note[enumerate]{
\item Any real-valued function can be decomposed into positive and negative parts.
\item This decomposition is unique and both parts are non-negative measurable functions.
\item A function is integrable if the integral of its absolute value is finite.
\item The space $L^1$ is a vector space with the integral as a linear functional.
\item Functions that differ on a null set are identified in $L^1$ (equivalence classes).
}

\begin{frame}{Properties of the Lebesgue integral}

\structure{Measure space $(S, \salg, \mu)$, integrable functions $f, g \in L^1(S, \salg, \mu)$}

\vskip 1em
\structure{Linearity:}
$$\int_S (\alpha f + \beta g) \, d\mu = \alpha \int_S f \, d\mu + \beta \int_S g \, d\mu, \quad \alpha, \beta \in \mathbb{R}$$

\vskip 0.5em
\structure{Monotonicity:} if $f \leq g$ a.e., then $\int f d\mu \leq \int g d\mu$

\vskip 0.5em
\structure{Triangle inequality:}
$$\left| \int_S f \, d\mu \right| \leq \int_S |f| \, d\mu$$

\vskip 0.5em
\structure{Additivity over domains:} for disjoint $A, B \in \salg$,
$$\int_{A \cup B} f \, d\mu = \int_A f \, d\mu + \int_B f \, d\mu$$

\end{frame}

\note[enumerate]{
\item Linearity is the most fundamental property - the integral is a linear functional on $L^1$.
\item Monotonicity allows us to compare integrals by comparing functions pointwise.
\item The triangle inequality is crucial for many estimates in analysis.
\item Domain additivity follows from the additivity of measures.
\item These properties make the Lebesgue integral a powerful tool for analysis.
}

\begin{frame}{Expectation as integral}

\structure{Probability space $(S, \salg, \mP)$, random variable $X: S \to \mathbb{R}$}

\vskip 1em
\structure{Expectation of $X$:}
$$\mathbb{E}[X] = \int_S X \, d\mP = \int_S X(s) \, \mP(ds)$$

if $X \in L^1(S, \salg, \mP)$, i.e., $\mathbb{E}[|X|] < \infty$

\vskip 1em
\structure{Alternative formulation via distribution $P_X$:}
$$\mathbb{E}[X] = \int_{\mathbb{R}} x \, P_X(dx)$$

push-forward: integrate over the range space with distribution measure

\vskip 1em
\structure{Properties:}
\begin{itemize}
  \item $\mathbb{E}[\alpha X + \beta Y] = \alpha \mathbb{E}[X] + \beta \mathbb{E}[Y]$ (linearity)
  \item if $X \leq Y$ a.s., then $\mathbb{E}[X] \leq \mathbb{E}[Y]$ (monotonicity)
  \item $|\mathbb{E}[X]| \leq \mathbb{E}[|X|]$ (triangle inequality)
\end{itemize}

\end{frame}

\note[enumerate]{
\item Expectation is simply the integral of a random variable with respect to the probability measure.
\item The notation $\mP(ds)$ emphasizes that we are integrating with respect to the measure $\mP$.
\item The alternative formulation using the distribution $P_X$ is often more practical for calculations.
\item All properties of the Lebesgue integral immediately apply to expectation.
\item A random variable is integrable if and only if its expectation is finite.
}

\begin{frame}{Comparison: Riemann vs Lebesgue integral}

\structure{Riemann integral} (partition domain into intervals)
\begin{itemize}
  \item approximate function by step functions based on domain partition
  \item sum: $\sum f(x_i^*) \Delta x_i$ where $x_i^*$ in $i$-th interval of width $\Delta x_i$
  \item limit as partition gets finer
  \item works for continuous functions and some discontinuous ones
\end{itemize}

\vskip 1em
\structure{Lebesgue integral} (partition range into levels)
\begin{itemize}
  \item approximate function by simple functions based on range partition
  \item sum: $\sum a_i \mu(\{s: f(s) \approx a_i\})$ where $a_i$ are values in range
  \item supremum over all simple function approximations
  \item works for all measurable functions
\end{itemize}

\vskip 1em
\structure{Relation:} Riemann integrable $\Rightarrow$ Lebesgue integrable with same value\\
\alert{But:} many Lebesgue integrable functions not Riemann integrable

\end{frame}

\note[enumerate]{
\item The key difference: Riemann partitions the domain (x-axis), Lebesgue partitions the range (y-axis).
\item Lebesgue's approach is more flexible and powerful, especially for limit theorems.
\item Example of Lebesgue but not Riemann integrable: indicator function of rationals in $[0,1]$.
\item For "nice" functions (continuous, piecewise continuous), both integrals coincide.
\item Lebesgue integration is essential for modern probability theory and functional analysis.
\item The Lebesgue integral has much better behavior under limits (see convergence theorems).
}

\begin{frame}{Example: discrete probability space}

\structure{Countable sample space $S = \{s_1, s_2, \ldots\}$, probability $\mP(\{s_i\}) = p_i$}

\vskip 1em
\structure{Random variable $X: S \to \mathbb{R}$, $X(s_i) = x_i$}

\vskip 1em
\structure{Expectation:}
$$\mathbb{E}[X] = \int_S X \, d\mP = \sum_{i=1}^{\infty} x_i p_i$$

\vskip 0.5em
if $\sum_{i=1}^{\infty} |x_i| p_i < \infty$ (absolute convergence)

\vskip 1em
\structure{Note:} 
\begin{itemize}
  \item counting measure: $\#(\{s_i\}) = 1$
  \item $\mP = \sum_{i} p_i \delta_{s_i}$ (sum of Dirac measures)
  \item integral reduces to weighted sum
\end{itemize}

\end{frame}

% \note[enumerate]{
% \item For discrete probability spaces, the Lebesgue integral reduces to the familiar weighted sum.
% \item The condition $\sum |x_i| p_i < \infty$ is necessary for the expectation to be well-defined.
% \item This shows that the Lebesgue integral generalizes the discrete}