\section{Transformations and push-forward}
\begin{frame}{Transformations of random variables}

\structure{Setup:} 
\begin{itemize}
\item $X$ - random variable with values in $T$ (e.g., $\mR^d$)
\item $P_X$ - distribution of $X$ on $T$
\item $g: T \to U$ - measurable function (transformation)
\item $Y = g(X)$ - transformed random variable with values in $U$
\end{itemize}

\vskip 1em
\structure{Question:} If we know $P_X$, what is the distribution $P_Y$ of $Y = g(X)$?

\vskip 1em
\structure{Examples in ML:}
\begin{itemize}
\item $X \sim \mathcal{N}(0, I)$ (simple), $g$ is neural network, $Y = g(X)$ (complex model distribution)
\item data transformation: $X$ is data, $g$ encodes to latent space
\end{itemize}

\end{frame}

\note[enumerate]
{
\item This is the fundamental question in generative modeling
\item We transform a simple distribution to get a complex one
\item The transformation $g$ is typically a neural network
\item We work directly with distributions on value spaces (as established in Section 2)
\item No need to refer back to abstract probability space $(S, \salg, \mP)$
}


\begin{frame}{Push-forward measure}

\structure{Answer:} push-forward of $P_X$ by $g$ gives distribution of $Y$

\vskip 0.5em
\structure{Definition:} $P_Y$ on $(U, \mathscr{U})$ defined by
\[
P_Y(C) = P_X(g^{-1}(C)) \quad \text{for } C \in \mathscr{U}
\]
where $g^{-1}(C) = \{x \in T: g(x) \in C\}$ is the pre-image

\vskip 1em
\centering
\scalebox{0.55}{
\begin{tikzpicture}[>=Stealth]
% LEFT: domain T with mesh and rectangular g^{-1}(C)
\begin{scope}[shift={(-5,0)}]
    \node at (-1,2) {$T$};
    % Mesh (vertical)
    \foreach \x in {-1.5,-1,...,1}
        \draw[gray!70] (\x,-2) -- (\x,2);
    % Mesh (horizontal)
    \foreach \y in {-1.5,-1,...,1.5}
        \draw[gray!70] (-2,\y) -- (1.5,\y);
    % Preimage region g^{-1}(C): ALIGNED WITH GRID
    \fill[blue!25,opacity=0.5] (-1,-1) rectangle (0,1);
    \draw[blue!60,thick] (-1,-1) rectangle (0,1);
    \node[blue!80] at (-0.5,0) {$g^{-1}(C)$};
\end{scope}
% Middle arrow: g
\draw[->,thick] (-3,0.5) -- (-1,0.5) node[midway,above] {$g: T \to U$};
\draw[<-,thick] (-3,-0.5) -- (-1,-0.5) node[midway,above] {preimage};
\node at (-2,-1.5) {\color{carnelian}$P_X(g^{-1}(C)) = P_Y(C)$};
% RIGHT: codomain U with warped mesh + image C
\begin{scope}[shift={(1,0)}]
    \node at (0,2) {$U$};
    % Warped vertical mesh lines
    \foreach \x in {-1.5,-1,...,1}
    \draw[gray!70,domain=-2:2,smooth,variable=\t]
        plot ({0.9*\x + 0.5*sin(\t r)}, {\t + 0.5*sin(\x r)});
    % Warped horizontal mesh lines
    \foreach \y in {-1.5,-1,...,1.5}
    \draw[gray!70,domain=-2:1.5,smooth,variable=\t]
        plot ({0.9*\t + 0.5*sin(\y r)}, {\y + 0.5*sin(\t r)});
    % Image region C
    \begin{scope}
        \fill[blue!25,opacity=0.5]
            plot[domain=-1:0,smooth,variable=\x]
                ({0.9*\x + 0.5*sin(-1 r)}, {-1 + 0.5*sin(\x r)})
            -- plot[domain=-1:1,smooth,variable=\y]
                ({0.9*0 + 0.5*sin(\y r)}, {\y + 0.5*sin(0 r)})
            -- plot[domain=0:-1,smooth,variable=\x]
                ({0.9*\x + 0.5*sin(1 r)}, {1 + 0.5*sin(\x r)})
            -- plot[domain=1:-1,smooth,variable=\y]
                ({0.9*(-1) + 0.5*sin(\y r)}, {\y + 0.5*sin(-1 r)})
            -- cycle;
        \draw[blue!60,thick]
            plot[domain=-1:0,smooth,variable=\x]
                ({0.9*\x + 0.5*sin(-1 r)}, {-1 + 0.5*sin(\x r)})
            -- plot[domain=-1:1,smooth,variable=\y]
                ({0.9*0 + 0.5*sin(\y r)}, {\y + 0.5*sin(0 r)})
            -- plot[domain=0:-1,smooth,variable=\x]
                ({0.9*\x + 0.5*sin(1 r)}, {1 + 0.5*sin(\x r)})
            -- plot[domain=1:-1,smooth,variable=\y]
                ({0.9*(-1) + 0.5*sin(\y r)}, {\y + 0.5*sin(-1 r)})
            -- cycle;
        \node[blue!80] at (-0.4,-0.1) {$C$};
    \end{scope}
\end{scope}
\end{tikzpicture}
}

\vskip 0.5em
\structure{Intuition:} probability flows through $g$ - probability of landing in $C$ equals probability of starting in pre-image

\end{frame}

\note[enumerate]
{
\item Push-forward is how probability transforms under functions
\item The pre-image $g^{-1}(C)$ is the set of points in $T$ that map to $C$
\item Diagram shows: regular grid in $T$ becomes warped in $U$, but probability is preserved
\item This is well-defined because $g$ is measurable: $g^{-1}(C) \in \mathscr{T}$ for all $C \in \mathscr{U}$
\item In ML: $g$ is neural network, $P_X$ is simple distribution, $P_Y$ is complex model distribution
}