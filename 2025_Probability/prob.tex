\begin{frame}{Definition of a Probability Space}
    A \textbf{probability space} is a mathematical framework for modeling random experiments
    and consists of three components:

    \begin{itemize}
        \item \textbf{Sample Space} (\(\Omega\)): 
        \begin{itemize}
            \item The set of all possible outcomes of a random experiment.
        \end{itemize}
        
        \item \textbf{Sigma-Algebra} (\(\mathcal{F}\)): 
        \begin{itemize}
            \item A collection of subsets of \(\Omega\), called events, that includes \(\Omega\) itself, is closed under complementation and countable unions.
        \end{itemize}
        
        \item \textbf{Probability Measure} (\(P\)): 
        \begin{itemize}
            \item A function \(P: \mathcal{F} \to [0, 1]\) that assigns a probability to each event in \(\mathcal{F}\).
            \item Satisfies: \(P(\Omega) = 1\) and is \(\sigma\)-additive.
        \end{itemize}
    \end{itemize}
    
    \textbf{Notation:} The probability space is denoted by \((\Omega, \mathcal{F}, P)\).
\end{frame}