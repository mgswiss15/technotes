%%%% Preamble %%%%
\documentclass[a4paper, oneside, 10pt]{article}

% wider text, smaller margins
\usepackage{a4wide}

% for coloring
\usepackage[dvipsnames]{xcolor}

% for index list
\usepackage{makeidx}
\makeindex
\newcommand{\addidx}[1]{\emph{#1}\index{\lowercase{#1}}}

% for better lists
\usepackage{enumitem}

% for images
\usepackage{graphicx}
\graphicspath{{./Pics/}}

% for hyperrefs
\usepackage{hyperref}

% for declaring math operators
\usepackage{amsthm}

% wrap text around figures
\usepackage{wrapfig}

% chagne format of section titles
\usepackage{titlesec}

% \titleformat*{\section}{\LARGE\bfseries}
% \titleformat*{\subsection}{\Large\bfseries}
% \titleformat*{\subsubsection}{\large\bfseries}
\titleformat*{\paragraph}{\bfseries\itshape}
% \titleformat*{\subparagraph}{\large\bfseries}

% nice text boxes
\usepackage{tcolorbox}
\newtcbox{\note}{on line, boxrule=0pt, boxsep=0pt, arc=0pt, outer arc=0pt, left=1pt, right=1pt, top=2pt, bottom=2pt, fontupper=\itshape}
% \newtcolorbox{notebox}{boxrule=0pt}
\newtcolorbox{notebox}[1][colback=black!5]{boxrule=0pt, #1}

% biblatex seetings for better referencing
\usepackage[backend=biber,
            citestyle=authoryear,
            bibstyle=numeric,
            block=space,
            backref=true,
            date=year,
            maxcitenames=5,
            mincitenames=1,
            url=false,
            doi=false,
            isbn=false,
            eprint=false]{biblatex}
\addbibresource{refs.bib}

%%% theorem environments
\theoremstyle{plain}
\newtheoremstyle{mytheoremstyle} % name
    {\topsep}{\topsep}{}{}%
    {\bfseries}% Theorem head font
    {}{.5em}%
    {\thmname{#1}\thmnumber{ #2}\thmnote{ (#3)}}  % Theorem head spec (can be left empty, meaning ‘normal’)
\theoremstyle{mytheoremstyle}
\newtheorem{theorem}{Theorem}[section]

\theoremstyle{definition}
\newtheoremstyle{mydefstyle} % name
    {\topsep}{\partopsep}{}{}%
    {\bfseries}% Theorem head font
    {}{.5em}%
    {\thmname{#1}\thmnumber{ #2}\thmnote{ (#3)}}  % Theorem head spec (can be left empty, meaning ‘normal’)
\theoremstyle{mydefstyle}
\newtheorem{definition}{Definition}[section]

% %%%%% NEW MATH DEFINITIONS %%%%%

\usepackage{amsmath,amsfonts,bm}

% Mark sections of captions for referring to divisions of figures
\newcommand{\figleft}{{\em (Left)}}
\newcommand{\figcenter}{{\em (Center)}}
\newcommand{\figright}{{\em (Right)}}
\newcommand{\figtop}{{\em (Top)}}
\newcommand{\figbottom}{{\em (Bottom)}}
\newcommand{\captiona}{{\em (a)}}
\newcommand{\captionb}{{\em (b)}}
\newcommand{\captionc}{{\em (c)}}
\newcommand{\captiond}{{\em (d)}}

% Highlight a newly defined term
\newcommand{\iterm}[1]{{\emph{#1}\index{#1}}}


% Figure reference, lower-case.
\def\figref#1{figure~\ref{#1}}
% Figure reference, capital. For start of sentence
\def\Figref#1{Figure~\ref{#1}}
\def\twofigref#1#2{figures \ref{#1} and \ref{#2}}
\def\quadfigref#1#2#3#4{figures \ref{#1}, \ref{#2}, \ref{#3} and \ref{#4}}
% Section reference, lower-case.
\def\secref#1{section~\ref{#1}}
% Section reference, capital.
\def\Secref#1{Section~\ref{#1}}
% Reference to two sections.
\def\twosecrefs#1#2{sections \ref{#1} and \ref{#2}}
% Reference to three sections.
\def\secrefs#1#2#3{sections \ref{#1}, \ref{#2} and \ref{#3}}
% % Reference to an equation, lower-case.
% \def\eqref#1{equation~\ref{#1}}
% % Reference to an equation, upper case
% \def\Eqref#1{Equation~\ref{#1}}
% % A raw reference to an equation---avoid using if possible
% \def\plaineqref#1{\ref{#1}}
% Reference to a chapter, lower-case.
\def\chapref#1{chapter~\ref{#1}}
% Reference to an equation, upper case.
\def\Chapref#1{Chapter~\ref{#1}}
% Reference to a range of chapters
\def\rangechapref#1#2{chapters\ref{#1}--\ref{#2}}
% Reference to an algorithm, lower-case.
\def\algref#1{algorithm~\ref{#1}}
% Reference to an algorithm, upper case.
\def\Algref#1{Algorithm~\ref{#1}}
\def\twoalgref#1#2{algorithms \ref{#1} and \ref{#2}}
\def\Twoalgref#1#2{Algorithms \ref{#1} and \ref{#2}}
% Reference to a part, lower case
\def\partref#1{part~\ref{#1}}
% Reference to a part, upper case
\def\Partref#1{Part~\ref{#1}}
\def\twopartref#1#2{parts \ref{#1} and \ref{#2}}

\def\ceil#1{\lceil #1 \rceil}
\def\floor#1{\lfloor #1 \rfloor}
\def\1{\bm{1}}
\newcommand{\train}{\mathcal{D}}
\newcommand{\valid}{\mathcal{D_{\mathrm{valid}}}}
\newcommand{\test}{\mathcal{D_{\mathrm{test}}}}

\def\eps{{\epsilon}}


% Random variables
\def\reta{{\textnormal{$\eta$}}}
\def\ra{{\textnormal{a}}}
\def\rb{{\textnormal{b}}}
\def\rc{{\textnormal{c}}}
\def\rd{{\textnormal{d}}}
\def\re{{\textnormal{e}}}
\def\rf{{\textnormal{f}}}
\def\rg{{\textnormal{g}}}
\def\rh{{\textnormal{h}}}
\def\ri{{\textnormal{i}}}
\def\rj{{\textnormal{j}}}
\def\rk{{\textnormal{k}}}
\def\rl{{\textnormal{l}}}
% rm is already a command, just don't name any random variables m
\def\rn{{\textnormal{n}}}
\def\ro{{\textnormal{o}}}
\def\rp{{\textnormal{p}}}
\def\rq{{\textnormal{q}}}
\def\rr{{\textnormal{r}}}
\def\rs{{\textnormal{s}}}
\def\rt{{\textnormal{t}}}
\def\ru{{\textnormal{u}}}
\def\rv{{\textnormal{v}}}
\def\rw{{\textnormal{w}}}
\def\rx{{\textnormal{x}}}
\def\ry{{\textnormal{y}}}
\def\rz{{\textnormal{z}}}

% Random vectors
\def\rvepsilon{{\mathbf{\epsilon}}}
\def\rvtheta{{\mathbf{\theta}}}
\def\rva{{\mathbf{a}}}
\def\rvb{{\mathbf{b}}}
\def\rvc{{\mathbf{c}}}
\def\rvd{{\mathbf{d}}}
\def\rve{{\mathbf{e}}}
\def\rvf{{\mathbf{f}}}
\def\rvg{{\mathbf{g}}}
\def\rvh{{\mathbf{h}}}
\def\rvu{{\mathbf{i}}}
\def\rvj{{\mathbf{j}}}
\def\rvk{{\mathbf{k}}}
\def\rvl{{\mathbf{l}}}
\def\rvm{{\mathbf{m}}}
\def\rvn{{\mathbf{n}}}
\def\rvo{{\mathbf{o}}}
\def\rvp{{\mathbf{p}}}
\def\rvq{{\mathbf{q}}}
\def\rvr{{\mathbf{r}}}
\def\rvs{{\mathbf{s}}}
\def\rvt{{\mathbf{t}}}
\def\rvu{{\mathbf{u}}}
\def\rvv{{\mathbf{v}}}
\def\rvw{{\mathbf{w}}}
\def\rvx{{\mathbf{x}}}
\def\rvy{{\mathbf{y}}}
\def\rvz{{\mathbf{z}}}

% Elements of random vectors
\def\erva{{\textnormal{a}}}
\def\ervb{{\textnormal{b}}}
\def\ervc{{\textnormal{c}}}
\def\ervd{{\textnormal{d}}}
\def\erve{{\textnormal{e}}}
\def\ervf{{\textnormal{f}}}
\def\ervg{{\textnormal{g}}}
\def\ervh{{\textnormal{h}}}
\def\ervi{{\textnormal{i}}}
\def\ervj{{\textnormal{j}}}
\def\ervk{{\textnormal{k}}}
\def\ervl{{\textnormal{l}}}
\def\ervm{{\textnormal{m}}}
\def\ervn{{\textnormal{n}}}
\def\ervo{{\textnormal{o}}}
\def\ervp{{\textnormal{p}}}
\def\ervq{{\textnormal{q}}}
\def\ervr{{\textnormal{r}}}
\def\ervs{{\textnormal{s}}}
\def\ervt{{\textnormal{t}}}
\def\ervu{{\textnormal{u}}}
\def\ervv{{\textnormal{v}}}
\def\ervw{{\textnormal{w}}}
\def\ervx{{\textnormal{x}}}
\def\ervy{{\textnormal{y}}}
\def\ervz{{\textnormal{z}}}

% Random matrices
\def\rmA{{\mathbf{A}}}
\def\rmB{{\mathbf{B}}}
\def\rmC{{\mathbf{C}}}
\def\rmD{{\mathbf{D}}}
\def\rmE{{\mathbf{E}}}
\def\rmF{{\mathbf{F}}}
\def\rmG{{\mathbf{G}}}
\def\rmH{{\mathbf{H}}}
\def\rmI{{\mathbf{I}}}
\def\rmJ{{\mathbf{J}}}
\def\rmK{{\mathbf{K}}}
\def\rmL{{\mathbf{L}}}
\def\rmM{{\mathbf{M}}}
\def\rmN{{\mathbf{N}}}
\def\rmO{{\mathbf{O}}}
\def\rmP{{\mathbf{P}}}
\def\rmQ{{\mathbf{Q}}}
\def\rmR{{\mathbf{R}}}
\def\rmS{{\mathbf{S}}}
\def\rmT{{\mathbf{T}}}
\def\rmU{{\mathbf{U}}}
\def\rmV{{\mathbf{V}}}
\def\rmW{{\mathbf{W}}}
\def\rmX{{\mathbf{X}}}
\def\rmY{{\mathbf{Y}}}
\def\rmZ{{\mathbf{Z}}}

% Elements of random matrices
\def\ermA{{\textnormal{A}}}
\def\ermB{{\textnormal{B}}}
\def\ermC{{\textnormal{C}}}
\def\ermD{{\textnormal{D}}}
\def\ermE{{\textnormal{E}}}
\def\ermF{{\textnormal{F}}}
\def\ermG{{\textnormal{G}}}
\def\ermH{{\textnormal{H}}}
\def\ermI{{\textnormal{I}}}
\def\ermJ{{\textnormal{J}}}
\def\ermK{{\textnormal{K}}}
\def\ermL{{\textnormal{L}}}
\def\ermM{{\textnormal{M}}}
\def\ermN{{\textnormal{N}}}
\def\ermO{{\textnormal{O}}}
\def\ermP{{\textnormal{P}}}
\def\ermQ{{\textnormal{Q}}}
\def\ermR{{\textnormal{R}}}
\def\ermS{{\textnormal{S}}}
\def\ermT{{\textnormal{T}}}
\def\ermU{{\textnormal{U}}}
\def\ermV{{\textnormal{V}}}
\def\ermW{{\textnormal{W}}}
\def\ermX{{\textnormal{X}}}
\def\ermY{{\textnormal{Y}}}
\def\ermZ{{\textnormal{Z}}}

% Vectors
\def\vzero{{\bm{0}}}
\def\vone{{\bm{1}}}
\def\vmu{{\bm{\mu}}}
\def\vtheta{{\bm{\theta}}}
\def\vphi{{\bm{\phi}}}
\def\va{{\bm{a}}}
\def\vb{{\bm{b}}}
\def\vc{{\bm{c}}}
\def\vd{{\bm{d}}}
\def\ve{{\bm{e}}}
\def\vf{{\bm{f}}}
\def\vg{{\bm{g}}}
\def\vh{{\bm{h}}}
\def\vi{{\bm{i}}}
\def\vj{{\bm{j}}}
\def\vk{{\bm{k}}}
\def\vl{{\bm{l}}}
\def\vm{{\bm{m}}}
\def\vn{{\bm{n}}}
\def\vo{{\bm{o}}}
\def\vp{{\bm{p}}}
\def\vq{{\bm{q}}}
\def\vr{{\bm{r}}}
\def\vs{{\bm{s}}}
\def\vt{{\bm{t}}}
\def\vu{{\bm{u}}}
\def\vv{{\bm{v}}}
\def\vw{{\bm{w}}}
\def\vx{{\bm{x}}}
\def\vy{{\bm{y}}}
\def\vz{{\bm{z}}}

% Elements of vectors
\def\evalpha{{\alpha}}
\def\evbeta{{\beta}}
\def\evepsilon{{\epsilon}}
\def\evlambda{{\lambda}}
\def\evomega{{\omega}}
\def\evmu{{\mu}}
\def\evpsi{{\psi}}
\def\evsigma{{\sigma}}
\def\evtheta{{\theta}}
\def\eva{{a}}
\def\evb{{b}}
\def\evc{{c}}
\def\evd{{d}}
\def\eve{{e}}
\def\evf{{f}}
\def\evg{{g}}
\def\evh{{h}}
\def\evi{{i}}
\def\evj{{j}}
\def\evk{{k}}
\def\evl{{l}}
\def\evm{{m}}
\def\evn{{n}}
\def\evo{{o}}
\def\evp{{p}}
\def\evq{{q}}
\def\evr{{r}}
\def\evs{{s}}
\def\evt{{t}}
\def\evu{{u}}
\def\evv{{v}}
\def\evw{{w}}
\def\evx{{x}}
\def\evy{{y}}
\def\evz{{z}}

% Matrix
\def\mA{{\bm{A}}}
\def\mB{{\bm{B}}}
\def\mC{{\bm{C}}}
\def\mD{{\bm{D}}}
\def\mE{{\bm{E}}}
\def\mF{{\bm{F}}}
\def\mG{{\bm{G}}}
\def\mH{{\bm{H}}}
\def\mI{{\bm{I}}}
\def\mJ{{\bm{J}}}
\def\mK{{\bm{K}}}
\def\mL{{\bm{L}}}
\def\mM{{\bm{M}}}
\def\mN{{\bm{N}}}
\def\mO{{\bm{O}}}
\def\mP{{\bm{P}}}
\def\mQ{{\bm{Q}}}
\def\mR{{\bm{R}}}
\def\mS{{\bm{S}}}
\def\mT{{\bm{T}}}
\def\mU{{\bm{U}}}
\def\mV{{\bm{V}}}
\def\mW{{\bm{W}}}
\def\mX{{\bm{X}}}
\def\mY{{\bm{Y}}}
\def\mZ{{\bm{Z}}}
\def\mBeta{{\bm{\beta}}}
\def\mPhi{{\bm{\Phi}}}
\def\mLambda{{\bm{\Lambda}}}
\def\mSigma{{\bm{\Sigma}}}

% Tensor
\DeclareMathAlphabet{\mathsfit}{\encodingdefault}{\sfdefault}{m}{sl}
\SetMathAlphabet{\mathsfit}{bold}{\encodingdefault}{\sfdefault}{bx}{n}
\newcommand{\tens}[1]{\bm{\mathsfit{#1}}}
\def\tA{{\tens{A}}}
\def\tB{{\tens{B}}}
\def\tC{{\tens{C}}}
\def\tD{{\tens{D}}}
\def\tE{{\tens{E}}}
\def\tF{{\tens{F}}}
\def\tG{{\tens{G}}}
\def\tH{{\tens{H}}}
\def\tI{{\tens{I}}}
\def\tJ{{\tens{J}}}
\def\tK{{\tens{K}}}
\def\tL{{\tens{L}}}
\def\tM{{\tens{M}}}
\def\tN{{\tens{N}}}
\def\tO{{\tens{O}}}
\def\tP{{\tens{P}}}
\def\tQ{{\tens{Q}}}
\def\tR{{\tens{R}}}
\def\tS{{\tens{S}}}
\def\tT{{\tens{T}}}
\def\tU{{\tens{U}}}
\def\tV{{\tens{V}}}
\def\tW{{\tens{W}}}
\def\tX{{\tens{X}}}
\def\tY{{\tens{Y}}}
\def\tZ{{\tens{Z}}}


% Graph
\def\gA{{\mathcal{A}}}
\def\gB{{\mathcal{B}}}
\def\gC{{\mathcal{C}}}
\def\gD{{\mathcal{D}}}
\def\gE{{\mathcal{E}}}
\def\gF{{\mathcal{F}}}
\def\gG{{\mathcal{G}}}
\def\gH{{\mathcal{H}}}
\def\gI{{\mathcal{I}}}
\def\gJ{{\mathcal{J}}}
\def\gK{{\mathcal{K}}}
\def\gL{{\mathcal{L}}}
\def\gM{{\mathcal{M}}}
\def\gN{{\mathcal{N}}}
\def\gO{{\mathcal{O}}}
\def\gP{{\mathcal{P}}}
\def\gQ{{\mathcal{Q}}}
\def\gR{{\mathcal{R}}}
\def\gS{{\mathcal{S}}}
\def\gT{{\mathcal{T}}}
\def\gU{{\mathcal{U}}}
\def\gV{{\mathcal{V}}}
\def\gW{{\mathcal{W}}}
\def\gX{{\mathcal{X}}}
\def\gY{{\mathcal{Y}}}
\def\gZ{{\mathcal{Z}}}

% Sets
\def\sA{{\mathbb{A}}}
\def\sB{{\mathbb{B}}}
\def\sC{{\mathbb{C}}}
\def\sD{{\mathbb{D}}}
% Don't use a set called E, because this would be the same as our symbol
% for expectation.
\def\sF{{\mathbb{F}}}
\def\sG{{\mathbb{G}}}
\def\sH{{\mathbb{H}}}
\def\sI{{\mathbb{I}}}
\def\sJ{{\mathbb{J}}}
\def\sK{{\mathbb{K}}}
\def\sL{{\mathbb{L}}}
\def\sM{{\mathbb{M}}}
\def\sN{{\mathbb{N}}}
\def\sO{{\mathbb{O}}}
\def\sP{{\mathbb{P}}}
\def\sQ{{\mathbb{Q}}}
\def\sR{{\mathbb{R}}}
\def\sS{{\mathbb{S}}}
\def\sT{{\mathbb{T}}}
\def\sU{{\mathbb{U}}}
\def\sV{{\mathbb{V}}}
\def\sW{{\mathbb{W}}}
\def\sX{{\mathbb{X}}}
\def\sY{{\mathbb{Y}}}
\def\sZ{{\mathbb{Z}}}

% Entries of a matrix
\def\emLambda{{\Lambda}}
\def\emA{{A}}
\def\emB{{B}}
\def\emC{{C}}
\def\emD{{D}}
\def\emE{{E}}
\def\emF{{F}}
\def\emG{{G}}
\def\emH{{H}}
\def\emI{{I}}
\def\emJ{{J}}
\def\emK{{K}}
\def\emL{{L}}
\def\emM{{M}}
\def\emN{{N}}
\def\emO{{O}}
\def\emP{{P}}
\def\emQ{{Q}}
\def\emR{{R}}
\def\emS{{S}}
\def\emT{{T}}
\def\emU{{U}}
\def\emV{{V}}
\def\emW{{W}}
\def\emX{{X}}
\def\emY{{Y}}
\def\emZ{{Z}}
\def\emSigma{{\Sigma}}

% entries of a tensor
% Same font as tensor, without \bm wrapper
\newcommand{\etens}[1]{\mathsfit{#1}}
\def\etLambda{{\etens{\Lambda}}}
\def\etA{{\etens{A}}}
\def\etB{{\etens{B}}}
\def\etC{{\etens{C}}}
\def\etD{{\etens{D}}}
\def\etE{{\etens{E}}}
\def\etF{{\etens{F}}}
\def\etG{{\etens{G}}}
\def\etH{{\etens{H}}}
\def\etI{{\etens{I}}}
\def\etJ{{\etens{J}}}
\def\etK{{\etens{K}}}
\def\etL{{\etens{L}}}
\def\etM{{\etens{M}}}
\def\etN{{\etens{N}}}
\def\etO{{\etens{O}}}
\def\etP{{\etens{P}}}
\def\etQ{{\etens{Q}}}
\def\etR{{\etens{R}}}
\def\etS{{\etens{S}}}
\def\etT{{\etens{T}}}
\def\etU{{\etens{U}}}
\def\etV{{\etens{V}}}
\def\etW{{\etens{W}}}
\def\etX{{\etens{X}}}
\def\etY{{\etens{Y}}}
\def\etZ{{\etens{Z}}}

% The true underlying data generating distribution
\newcommand{\pdata}{p_{\rm{data}}}
% The empirical distribution defined by the training set
\newcommand{\ptrain}{\hat{p}_{\rm{data}}}
\newcommand{\Ptrain}{\hat{P}_{\rm{data}}}
% The model distribution
\newcommand{\pmodel}{p_{\rm{model}}}
\newcommand{\Pmodel}{P_{\rm{model}}}
\newcommand{\ptildemodel}{\tilde{p}_{\rm{model}}}
% Stochastic autoencoder distributions
\newcommand{\pencode}{p_{\rm{encoder}}}
\newcommand{\pdecode}{p_{\rm{decoder}}}
\newcommand{\precons}{p_{\rm{reconstruct}}}

\newcommand{\laplace}{\mathrm{Laplace}} % Laplace distribution

\newcommand{\E}{\mathbb{E}}
\newcommand{\Ls}{\mathcal{L}}
\newcommand{\R}{\mathbb{R}}
\newcommand{\emp}{\tilde{p}}
\newcommand{\lr}{\alpha}
\newcommand{\reg}{\lambda}
\newcommand{\rect}{\mathrm{rectifier}}
\newcommand{\softmax}{\mathrm{softmax}}
\newcommand{\sigmoid}{\sigma}
\newcommand{\softplus}{\zeta}
\newcommand{\KL}{D_{\mathrm{KL}}}
\newcommand{\Var}{\mathrm{Var}}
\newcommand{\standarderror}{\mathrm{SE}}
\newcommand{\Cov}{\mathrm{Cov}}
\newcommand{\logb}{\log_2}
% Wolfram Mathworld says $L^2$ is for function spaces and $\ell^2$ is for vectors
% But then they seem to use $L^2$ for vectors throughout the site, and so does
% wikipedia.
\newcommand{\normlzero}{L^0}
\newcommand{\normlone}{L^1}
\newcommand{\normltwo}{L^2}
\newcommand{\normlp}{L^p}
\newcommand{\normmax}{L^\infty}

\newcommand{\parents}{Pa} % See usage in notation.tex. Chosen to match Daphne's book.

\DeclareMathOperator*{\argmax}{arg\,max}
\DeclareMathOperator*{\argmin}{arg\,min}

\DeclareMathOperator{\sign}{sign}
\DeclareMathOperator{\Tr}{Tr}
\let\ab\allowbreak

% PACKAGES
\usepackage{xcolor}
\usepackage{import}

%%%%% NEW MATH DEFINITIONS %%%%%

\usepackage{amsmath,amsfonts,bm}

% Mark sections of captions for referring to divisions of figures
\newcommand{\figleft}{{\em (Left)}}
\newcommand{\figcenter}{{\em (Center)}}
\newcommand{\figright}{{\em (Right)}}
\newcommand{\figtop}{{\em (Top)}}
\newcommand{\figbottom}{{\em (Bottom)}}
\newcommand{\captiona}{{\em (a)}}
\newcommand{\captionb}{{\em (b)}}
\newcommand{\captionc}{{\em (c)}}
\newcommand{\captiond}{{\em (d)}}

% Highlight a newly defined term
\newcommand{\iterm}[1]{{\emph{#1}\index{#1}}}


% Figure reference, lower-case.
\def\figref#1{figure~\ref{#1}}
% Figure reference, capital. For start of sentence
\def\Figref#1{Figure~\ref{#1}}
\def\twofigref#1#2{figures \ref{#1} and \ref{#2}}
\def\quadfigref#1#2#3#4{figures \ref{#1}, \ref{#2}, \ref{#3} and \ref{#4}}
% Section reference, lower-case.
\def\secref#1{section~\ref{#1}}
% Section reference, capital.
\def\Secref#1{Section~\ref{#1}}
% Reference to two sections.
\def\twosecrefs#1#2{sections \ref{#1} and \ref{#2}}
% Reference to three sections.
\def\secrefs#1#2#3{sections \ref{#1}, \ref{#2} and \ref{#3}}
% % Reference to an equation, lower-case.
% \def\eqref#1{equation~\ref{#1}}
% % Reference to an equation, upper case
% \def\Eqref#1{Equation~\ref{#1}}
% % A raw reference to an equation---avoid using if possible
% \def\plaineqref#1{\ref{#1}}
% Reference to a chapter, lower-case.
\def\chapref#1{chapter~\ref{#1}}
% Reference to an equation, upper case.
\def\Chapref#1{Chapter~\ref{#1}}
% Reference to a range of chapters
\def\rangechapref#1#2{chapters\ref{#1}--\ref{#2}}
% Reference to an algorithm, lower-case.
\def\algref#1{algorithm~\ref{#1}}
% Reference to an algorithm, upper case.
\def\Algref#1{Algorithm~\ref{#1}}
\def\twoalgref#1#2{algorithms \ref{#1} and \ref{#2}}
\def\Twoalgref#1#2{Algorithms \ref{#1} and \ref{#2}}
% Reference to a part, lower case
\def\partref#1{part~\ref{#1}}
% Reference to a part, upper case
\def\Partref#1{Part~\ref{#1}}
\def\twopartref#1#2{parts \ref{#1} and \ref{#2}}

\def\ceil#1{\lceil #1 \rceil}
\def\floor#1{\lfloor #1 \rfloor}
\def\1{\bm{1}}
\newcommand{\train}{\mathcal{D}}
\newcommand{\valid}{\mathcal{D_{\mathrm{valid}}}}
\newcommand{\test}{\mathcal{D_{\mathrm{test}}}}

\def\eps{{\epsilon}}


% Random variables
\def\reta{{\textnormal{$\eta$}}}
\def\ra{{\textnormal{a}}}
\def\rb{{\textnormal{b}}}
\def\rc{{\textnormal{c}}}
\def\rd{{\textnormal{d}}}
\def\re{{\textnormal{e}}}
\def\rf{{\textnormal{f}}}
\def\rg{{\textnormal{g}}}
\def\rh{{\textnormal{h}}}
\def\ri{{\textnormal{i}}}
\def\rj{{\textnormal{j}}}
\def\rk{{\textnormal{k}}}
\def\rl{{\textnormal{l}}}
% rm is already a command, just don't name any random variables m
\def\rn{{\textnormal{n}}}
\def\ro{{\textnormal{o}}}
\def\rp{{\textnormal{p}}}
\def\rq{{\textnormal{q}}}
\def\rr{{\textnormal{r}}}
\def\rs{{\textnormal{s}}}
\def\rt{{\textnormal{t}}}
\def\ru{{\textnormal{u}}}
\def\rv{{\textnormal{v}}}
\def\rw{{\textnormal{w}}}
\def\rx{{\textnormal{x}}}
\def\ry{{\textnormal{y}}}
\def\rz{{\textnormal{z}}}

% Random vectors
\def\rvepsilon{{\mathbf{\epsilon}}}
\def\rvtheta{{\mathbf{\theta}}}
\def\rva{{\mathbf{a}}}
\def\rvb{{\mathbf{b}}}
\def\rvc{{\mathbf{c}}}
\def\rvd{{\mathbf{d}}}
\def\rve{{\mathbf{e}}}
\def\rvf{{\mathbf{f}}}
\def\rvg{{\mathbf{g}}}
\def\rvh{{\mathbf{h}}}
\def\rvu{{\mathbf{i}}}
\def\rvj{{\mathbf{j}}}
\def\rvk{{\mathbf{k}}}
\def\rvl{{\mathbf{l}}}
\def\rvm{{\mathbf{m}}}
\def\rvn{{\mathbf{n}}}
\def\rvo{{\mathbf{o}}}
\def\rvp{{\mathbf{p}}}
\def\rvq{{\mathbf{q}}}
\def\rvr{{\mathbf{r}}}
\def\rvs{{\mathbf{s}}}
\def\rvt{{\mathbf{t}}}
\def\rvu{{\mathbf{u}}}
\def\rvv{{\mathbf{v}}}
\def\rvw{{\mathbf{w}}}
\def\rvx{{\mathbf{x}}}
\def\rvy{{\mathbf{y}}}
\def\rvz{{\mathbf{z}}}

% Elements of random vectors
\def\erva{{\textnormal{a}}}
\def\ervb{{\textnormal{b}}}
\def\ervc{{\textnormal{c}}}
\def\ervd{{\textnormal{d}}}
\def\erve{{\textnormal{e}}}
\def\ervf{{\textnormal{f}}}
\def\ervg{{\textnormal{g}}}
\def\ervh{{\textnormal{h}}}
\def\ervi{{\textnormal{i}}}
\def\ervj{{\textnormal{j}}}
\def\ervk{{\textnormal{k}}}
\def\ervl{{\textnormal{l}}}
\def\ervm{{\textnormal{m}}}
\def\ervn{{\textnormal{n}}}
\def\ervo{{\textnormal{o}}}
\def\ervp{{\textnormal{p}}}
\def\ervq{{\textnormal{q}}}
\def\ervr{{\textnormal{r}}}
\def\ervs{{\textnormal{s}}}
\def\ervt{{\textnormal{t}}}
\def\ervu{{\textnormal{u}}}
\def\ervv{{\textnormal{v}}}
\def\ervw{{\textnormal{w}}}
\def\ervx{{\textnormal{x}}}
\def\ervy{{\textnormal{y}}}
\def\ervz{{\textnormal{z}}}

% Random matrices
\def\rmA{{\mathbf{A}}}
\def\rmB{{\mathbf{B}}}
\def\rmC{{\mathbf{C}}}
\def\rmD{{\mathbf{D}}}
\def\rmE{{\mathbf{E}}}
\def\rmF{{\mathbf{F}}}
\def\rmG{{\mathbf{G}}}
\def\rmH{{\mathbf{H}}}
\def\rmI{{\mathbf{I}}}
\def\rmJ{{\mathbf{J}}}
\def\rmK{{\mathbf{K}}}
\def\rmL{{\mathbf{L}}}
\def\rmM{{\mathbf{M}}}
\def\rmN{{\mathbf{N}}}
\def\rmO{{\mathbf{O}}}
\def\rmP{{\mathbf{P}}}
\def\rmQ{{\mathbf{Q}}}
\def\rmR{{\mathbf{R}}}
\def\rmS{{\mathbf{S}}}
\def\rmT{{\mathbf{T}}}
\def\rmU{{\mathbf{U}}}
\def\rmV{{\mathbf{V}}}
\def\rmW{{\mathbf{W}}}
\def\rmX{{\mathbf{X}}}
\def\rmY{{\mathbf{Y}}}
\def\rmZ{{\mathbf{Z}}}

% Elements of random matrices
\def\ermA{{\textnormal{A}}}
\def\ermB{{\textnormal{B}}}
\def\ermC{{\textnormal{C}}}
\def\ermD{{\textnormal{D}}}
\def\ermE{{\textnormal{E}}}
\def\ermF{{\textnormal{F}}}
\def\ermG{{\textnormal{G}}}
\def\ermH{{\textnormal{H}}}
\def\ermI{{\textnormal{I}}}
\def\ermJ{{\textnormal{J}}}
\def\ermK{{\textnormal{K}}}
\def\ermL{{\textnormal{L}}}
\def\ermM{{\textnormal{M}}}
\def\ermN{{\textnormal{N}}}
\def\ermO{{\textnormal{O}}}
\def\ermP{{\textnormal{P}}}
\def\ermQ{{\textnormal{Q}}}
\def\ermR{{\textnormal{R}}}
\def\ermS{{\textnormal{S}}}
\def\ermT{{\textnormal{T}}}
\def\ermU{{\textnormal{U}}}
\def\ermV{{\textnormal{V}}}
\def\ermW{{\textnormal{W}}}
\def\ermX{{\textnormal{X}}}
\def\ermY{{\textnormal{Y}}}
\def\ermZ{{\textnormal{Z}}}

% Vectors
\def\vzero{{\bm{0}}}
\def\vone{{\bm{1}}}
\def\vmu{{\bm{\mu}}}
\def\vtheta{{\bm{\theta}}}
\def\vphi{{\bm{\phi}}}
\def\va{{\bm{a}}}
\def\vb{{\bm{b}}}
\def\vc{{\bm{c}}}
\def\vd{{\bm{d}}}
\def\ve{{\bm{e}}}
\def\vf{{\bm{f}}}
\def\vg{{\bm{g}}}
\def\vh{{\bm{h}}}
\def\vi{{\bm{i}}}
\def\vj{{\bm{j}}}
\def\vk{{\bm{k}}}
\def\vl{{\bm{l}}}
\def\vm{{\bm{m}}}
\def\vn{{\bm{n}}}
\def\vo{{\bm{o}}}
\def\vp{{\bm{p}}}
\def\vq{{\bm{q}}}
\def\vr{{\bm{r}}}
\def\vs{{\bm{s}}}
\def\vt{{\bm{t}}}
\def\vu{{\bm{u}}}
\def\vv{{\bm{v}}}
\def\vw{{\bm{w}}}
\def\vx{{\bm{x}}}
\def\vy{{\bm{y}}}
\def\vz{{\bm{z}}}

% Elements of vectors
\def\evalpha{{\alpha}}
\def\evbeta{{\beta}}
\def\evepsilon{{\epsilon}}
\def\evlambda{{\lambda}}
\def\evomega{{\omega}}
\def\evmu{{\mu}}
\def\evpsi{{\psi}}
\def\evsigma{{\sigma}}
\def\evtheta{{\theta}}
\def\eva{{a}}
\def\evb{{b}}
\def\evc{{c}}
\def\evd{{d}}
\def\eve{{e}}
\def\evf{{f}}
\def\evg{{g}}
\def\evh{{h}}
\def\evi{{i}}
\def\evj{{j}}
\def\evk{{k}}
\def\evl{{l}}
\def\evm{{m}}
\def\evn{{n}}
\def\evo{{o}}
\def\evp{{p}}
\def\evq{{q}}
\def\evr{{r}}
\def\evs{{s}}
\def\evt{{t}}
\def\evu{{u}}
\def\evv{{v}}
\def\evw{{w}}
\def\evx{{x}}
\def\evy{{y}}
\def\evz{{z}}

% Matrix
\def\mA{{\bm{A}}}
\def\mB{{\bm{B}}}
\def\mC{{\bm{C}}}
\def\mD{{\bm{D}}}
\def\mE{{\bm{E}}}
\def\mF{{\bm{F}}}
\def\mG{{\bm{G}}}
\def\mH{{\bm{H}}}
\def\mI{{\bm{I}}}
\def\mJ{{\bm{J}}}
\def\mK{{\bm{K}}}
\def\mL{{\bm{L}}}
\def\mM{{\bm{M}}}
\def\mN{{\bm{N}}}
\def\mO{{\bm{O}}}
\def\mP{{\bm{P}}}
\def\mQ{{\bm{Q}}}
\def\mR{{\bm{R}}}
\def\mS{{\bm{S}}}
\def\mT{{\bm{T}}}
\def\mU{{\bm{U}}}
\def\mV{{\bm{V}}}
\def\mW{{\bm{W}}}
\def\mX{{\bm{X}}}
\def\mY{{\bm{Y}}}
\def\mZ{{\bm{Z}}}
\def\mBeta{{\bm{\beta}}}
\def\mPhi{{\bm{\Phi}}}
\def\mLambda{{\bm{\Lambda}}}
\def\mSigma{{\bm{\Sigma}}}

% Tensor
\DeclareMathAlphabet{\mathsfit}{\encodingdefault}{\sfdefault}{m}{sl}
\SetMathAlphabet{\mathsfit}{bold}{\encodingdefault}{\sfdefault}{bx}{n}
\newcommand{\tens}[1]{\bm{\mathsfit{#1}}}
\def\tA{{\tens{A}}}
\def\tB{{\tens{B}}}
\def\tC{{\tens{C}}}
\def\tD{{\tens{D}}}
\def\tE{{\tens{E}}}
\def\tF{{\tens{F}}}
\def\tG{{\tens{G}}}
\def\tH{{\tens{H}}}
\def\tI{{\tens{I}}}
\def\tJ{{\tens{J}}}
\def\tK{{\tens{K}}}
\def\tL{{\tens{L}}}
\def\tM{{\tens{M}}}
\def\tN{{\tens{N}}}
\def\tO{{\tens{O}}}
\def\tP{{\tens{P}}}
\def\tQ{{\tens{Q}}}
\def\tR{{\tens{R}}}
\def\tS{{\tens{S}}}
\def\tT{{\tens{T}}}
\def\tU{{\tens{U}}}
\def\tV{{\tens{V}}}
\def\tW{{\tens{W}}}
\def\tX{{\tens{X}}}
\def\tY{{\tens{Y}}}
\def\tZ{{\tens{Z}}}


% Graph
\def\gA{{\mathcal{A}}}
\def\gB{{\mathcal{B}}}
\def\gC{{\mathcal{C}}}
\def\gD{{\mathcal{D}}}
\def\gE{{\mathcal{E}}}
\def\gF{{\mathcal{F}}}
\def\gG{{\mathcal{G}}}
\def\gH{{\mathcal{H}}}
\def\gI{{\mathcal{I}}}
\def\gJ{{\mathcal{J}}}
\def\gK{{\mathcal{K}}}
\def\gL{{\mathcal{L}}}
\def\gM{{\mathcal{M}}}
\def\gN{{\mathcal{N}}}
\def\gO{{\mathcal{O}}}
\def\gP{{\mathcal{P}}}
\def\gQ{{\mathcal{Q}}}
\def\gR{{\mathcal{R}}}
\def\gS{{\mathcal{S}}}
\def\gT{{\mathcal{T}}}
\def\gU{{\mathcal{U}}}
\def\gV{{\mathcal{V}}}
\def\gW{{\mathcal{W}}}
\def\gX{{\mathcal{X}}}
\def\gY{{\mathcal{Y}}}
\def\gZ{{\mathcal{Z}}}

% Sets
\def\sA{{\mathbb{A}}}
\def\sB{{\mathbb{B}}}
\def\sC{{\mathbb{C}}}
\def\sD{{\mathbb{D}}}
% Don't use a set called E, because this would be the same as our symbol
% for expectation.
\def\sF{{\mathbb{F}}}
\def\sG{{\mathbb{G}}}
\def\sH{{\mathbb{H}}}
\def\sI{{\mathbb{I}}}
\def\sJ{{\mathbb{J}}}
\def\sK{{\mathbb{K}}}
\def\sL{{\mathbb{L}}}
\def\sM{{\mathbb{M}}}
\def\sN{{\mathbb{N}}}
\def\sO{{\mathbb{O}}}
\def\sP{{\mathbb{P}}}
\def\sQ{{\mathbb{Q}}}
\def\sR{{\mathbb{R}}}
\def\sS{{\mathbb{S}}}
\def\sT{{\mathbb{T}}}
\def\sU{{\mathbb{U}}}
\def\sV{{\mathbb{V}}}
\def\sW{{\mathbb{W}}}
\def\sX{{\mathbb{X}}}
\def\sY{{\mathbb{Y}}}
\def\sZ{{\mathbb{Z}}}

% Entries of a matrix
\def\emLambda{{\Lambda}}
\def\emA{{A}}
\def\emB{{B}}
\def\emC{{C}}
\def\emD{{D}}
\def\emE{{E}}
\def\emF{{F}}
\def\emG{{G}}
\def\emH{{H}}
\def\emI{{I}}
\def\emJ{{J}}
\def\emK{{K}}
\def\emL{{L}}
\def\emM{{M}}
\def\emN{{N}}
\def\emO{{O}}
\def\emP{{P}}
\def\emQ{{Q}}
\def\emR{{R}}
\def\emS{{S}}
\def\emT{{T}}
\def\emU{{U}}
\def\emV{{V}}
\def\emW{{W}}
\def\emX{{X}}
\def\emY{{Y}}
\def\emZ{{Z}}
\def\emSigma{{\Sigma}}

% entries of a tensor
% Same font as tensor, without \bm wrapper
\newcommand{\etens}[1]{\mathsfit{#1}}
\def\etLambda{{\etens{\Lambda}}}
\def\etA{{\etens{A}}}
\def\etB{{\etens{B}}}
\def\etC{{\etens{C}}}
\def\etD{{\etens{D}}}
\def\etE{{\etens{E}}}
\def\etF{{\etens{F}}}
\def\etG{{\etens{G}}}
\def\etH{{\etens{H}}}
\def\etI{{\etens{I}}}
\def\etJ{{\etens{J}}}
\def\etK{{\etens{K}}}
\def\etL{{\etens{L}}}
\def\etM{{\etens{M}}}
\def\etN{{\etens{N}}}
\def\etO{{\etens{O}}}
\def\etP{{\etens{P}}}
\def\etQ{{\etens{Q}}}
\def\etR{{\etens{R}}}
\def\etS{{\etens{S}}}
\def\etT{{\etens{T}}}
\def\etU{{\etens{U}}}
\def\etV{{\etens{V}}}
\def\etW{{\etens{W}}}
\def\etX{{\etens{X}}}
\def\etY{{\etens{Y}}}
\def\etZ{{\etens{Z}}}

% The true underlying data generating distribution
\newcommand{\pdata}{p_{\rm{data}}}
% The empirical distribution defined by the training set
\newcommand{\ptrain}{\hat{p}_{\rm{data}}}
\newcommand{\Ptrain}{\hat{P}_{\rm{data}}}
% The model distribution
\newcommand{\pmodel}{p_{\rm{model}}}
\newcommand{\Pmodel}{P_{\rm{model}}}
\newcommand{\ptildemodel}{\tilde{p}_{\rm{model}}}
% Stochastic autoencoder distributions
\newcommand{\pencode}{p_{\rm{encoder}}}
\newcommand{\pdecode}{p_{\rm{decoder}}}
\newcommand{\precons}{p_{\rm{reconstruct}}}

\newcommand{\laplace}{\mathrm{Laplace}} % Laplace distribution

\newcommand{\E}{\mathbb{E}}
\newcommand{\Ls}{\mathcal{L}}
\newcommand{\R}{\mathbb{R}}
\newcommand{\emp}{\tilde{p}}
\newcommand{\lr}{\alpha}
\newcommand{\reg}{\lambda}
\newcommand{\rect}{\mathrm{rectifier}}
\newcommand{\softmax}{\mathrm{softmax}}
\newcommand{\sigmoid}{\sigma}
\newcommand{\softplus}{\zeta}
\newcommand{\KL}{D_{\mathrm{KL}}}
\newcommand{\Var}{\mathrm{Var}}
\newcommand{\standarderror}{\mathrm{SE}}
\newcommand{\Cov}{\mathrm{Cov}}
\newcommand{\logb}{\log_2}
% Wolfram Mathworld says $L^2$ is for function spaces and $\ell^2$ is for vectors
% But then they seem to use $L^2$ for vectors throughout the site, and so does
% wikipedia.
\newcommand{\normlzero}{L^0}
\newcommand{\normlone}{L^1}
\newcommand{\normltwo}{L^2}
\newcommand{\normlp}{L^p}
\newcommand{\normmax}{L^\infty}

\newcommand{\parents}{Pa} % See usage in notation.tex. Chosen to match Daphne's book.

\DeclareMathOperator*{\argmax}{arg\,max}
\DeclareMathOperator*{\argmin}{arg\,min}

\DeclareMathOperator{\sign}{sign}
\DeclareMathOperator{\Tr}{Tr}
\let\ab\allowbreak


% COLORS, SYMBOLS
\newcommand{\cmark}{\ding{51}}
\newcommand{\xmark}{\ding{55}}
\newcommand{\colorred}{\color{Red}}
\newcommand{\colorgreen}{\color{ForestGreen}}
\newcommand{\colorgrey}{\color{darkgray}}
\newcommand{\colorblack}{\color{Black}}


% MATH
\newcommand{\norm}[1]{\left\lVert#1\right\rVert}%
\newcommand{\ovx}{\bm{x}^*}
\newcommand{\ovy}{y^*}
\newcommand{\yt}{y_t}
\newcommand{\hvx}{\bm{\hat{x}}}
\newcommand{\hvX}{\bm{\hat{X}}}
\newcommand{\pgen}{p_{\rm{gen}}}
\newcommand{\pcl}{p_{\rm{cl}}}
\newcommand{\rvX}{\bm{X}}
\newcommand{\rvY}{\bm{Y}}
\newcommand{\scg}{s_{cl}} % Scaling classifier guidance
\newcommand{\cg}{\mu_{cg}} %C lassifier guidance
\newcommand{\ssg}{s_d} % Scaling structural guidance
\newcommand{\sg}{\mu_{sg}} % Structural guidance

% Diffusion
\newcommand{\pt}{p_{\theta}} % diffusion reverse process
\newcommand{\pf}{p_{\phi}} % classifier
\newcommand{\mt}{\vmu_{\theta}} % diffusion reverse process mean
\newcommand{\st}{\mSigma_{\theta}} % diffusion reverse process covariance
\newcommand{\dN}{\mathcal{N}} % Normal distribution
\newcommand{\veta}{\bm{\eta}} % Exponential family natural parameters
\newcommand{\Ts}{\mathrm{T}} % Transpose
\newcommand{\vvec}{\mathrm{vec}} % vectorization operator
\newcommand{\amint}{\argmin_{\theta}}
\newcommand{\epst}{\rvepsilon_{\theta}}
\newcommand{\ut}{\vu_{\theta}}
\newcommand{\gradxt}{\nabla_{\rvx_t}}

%%% new shorthand commands
\newcommand{\nn}{\nonumber \\}%

\newcommand{\tldr}{\textbf{TLDR: }}
\newcommand{\concl}{\textbf{Concluding remarks: }}


\begin{document}

% where pics are
\graphicspath{ {./Pics/} }

% supress paragraph idents
\setlength{\parindent}{0pt}
\setlength{\parskip}{1ex plus 0.5ex minus 0.2ex}

% Number equations within a section
\numberwithin{equation}{section}

{\large Magda's technical notes on diffusion}

{\hfill Last update: \today}

This is another set of my technical notes on various ML topics.
I started writing the 1st set when beginning my PhD, the 2nd set when starting my PostDoc, the 3d when starting as a professor and I have discovered, that forcing myself to take time and write these is extremely useful.
All of the technical notes are available in my GitHub repo \url{https://github.com/mgswiss15/technotes}.

The general purpose of the notes is to help me understand better the selected topics by re-explaining (\emph{re-} because these have been explained elsewhere many times), and to have a reference and possibly reusable material for later.

This is a working document not meant to be polished. There may be typos and other editing errors. 
Technical errors mean that I didn't quite understand something which I unfortunately cannot rule out.


% Print table of contents
\tableofcontents

% !TEX root = main.tex
\clearpage

\section{Basics of diffusion models}\label{sec:diffusion_basics}

\begin{notebox}
\textbf{Using:} 
\fullcite{ho_denoising_2020} \\
\fullcite{weng_what_2021}.
\end{notebox}

\subsection{Reverse process}

We have true data coming from an unknown underlying distribution $\rvx_0 \sim q(\rvx_0)$.

We assume a latent variable model $\pt(\rvx_0) \approx q(\rvx_0)$ approximating the true distribution $\pt(\rvx_0) = \int \pt(\rvx_0 | \rvx_{1:T}) \pt(\rvx_{1:T}) \, d\rvx_{1:T}$.
For this we assume a learned \addidx{prior} $\pt(\rvx_{1:T})$ with Markov chain with Gaussian transitions (the means and variances are learned):
\begin{align*}
    \pt(\rvx_{1:T}) &= p(\rvx_T) \prod_{t=2}^T \pt(\rvx_{t-1} | \rvx_t) \\
    \pt(\rvx_{t-1} | \rvx_t) &= \dN(\rvx_{t-1}; \mt(\rvx_t, t), \st(\rvx_t, t)) \\
    p(\rvx_T) & = \dN(\rvx_T; \mathbf{0}, \mI) \enspace .
\end{align*}
The complete joint distribution $\pt(\rvx_{0:T})$ is called the \addidx{reverse process}.

\subsection{Variational bound}

\paragraph{Follow the logic of importance sampling of VAE (see technical notes 2019):}
We could start maximizing the likelihood $\pt(\rvx_0)$ directly from 
\begin{equation}
    \pt(\rvx_0) = \int \pt(\rvx_{1:T}) \pt(\rvx_0 | \rvx_{1:T}) \, d\rvx_{1:T}
\end{equation}
by sampling from the prior $\pt(\rvx_{1:T})$.
Same as always, this would take very long cause the prior samples won't be very informative for the true data and won't give enough information for the training.

We could instead sample from the posterior $\pt(\rvx_{1:T} | \rvx_0) = \frac{\pt(\rvx_{0:T})}{\pt(\rvx_0)}$ using the \addidx{importance sampling} which should be more informative
\begin{equation}
    \pt(\rvx_0) = \int \pt(\rvx_{1:T} | \rvx_0) \frac{\pt(\rvx_{1:T}) \pt(\rvx_0 | \rvx_{1:T})}{\pt(\rvx_{1:T} | \rvx_0)} \, d\rvx_{1:T}
\end{equation}
The problem as always is that the posterior is intractable due to the unknown evidence $\pt(\rvx_0)$.

Hence we need an approximation instead $q(\rvx_{1:T} | \rvx_0) \approx \pt(\rvx_{1:T} | \rvx_0)$ so that
\begin{equation}
    \pt(\rvx_0) = \int q(\rvx_{1:T} | \rvx_0) \frac{\pt(\rvx_{1:T}) \pt(\rvx_0 | \rvx_{1:T})}{q(\rvx_{1:T} | \rvx_0)} \, d\rvx_{1:T}
\end{equation}

This indeed is very similar to a VAE when we indicate the whole sequence $\rvx_{1:T}$ as a single latent variable $\rvz$.

We can now maximize the model likelihood $\pt(\rvx_0)$ by sampling the latent variable $\rvx_{1:T} \sim q(\rvx_{1:T} | \rvx_0)$ from the approximate posteriors which can be seen as the \addidx{encoder}.
We also have the learned \addidx{decoder} $\pt(\rvx_0 | \rvx_{1:T})$ and a prior $\pt(\rvx_{1:T})$ which in this case is learned.

We could train the model from the classical variational bound on the log likelihood
\begin{equation}
    \log \pt(\rvx_0) 
    = 
    \log \int q(\rvx_{1:T} | \rvx_0) \frac{\pt(\rvx_{0:T})}{q(\rvx_{1:T} | \rvx_0)} \, d\rvx_{1:T}
    \geq 
    \int q(\rvx_{1:T} | \rvx_0) \log \frac{\pt(\rvx_{0:T})}{q(\rvx_{1:T} | \rvx_0)} \, d\rvx_{1:T}
\end{equation}

\paragraph{VAE-like minimization problem:} 
Classically we minimize the negative log likelihood.
The objective is thus the minimization of the variational bound
\begin{align}\label{eq:VAE_loss}
    \Ls(\rvx_0) & = \E_{q(\rvx_{1:T} | \rvx_0)} - \log \frac{\pt(\rvx_{0:T})}{q(\rvx_{1:T} | \rvx_0)} \nn
     & = \E_{q(\rvx_{1:T} | \rvx_0)} \log \frac{q(\rvx_{1:T} | \rvx_0)}{\pt(\rvx_{0:T-1}|\rvx_T) \pt(\rvx_T)} \nn
     & = \E_{q(\rvx_{1:T} | \rvx_0)} \Big[- \log \pt(\rvx_T) - \log \prod_{t=1}^T \pt(\rvx_{t-1}|\rvx_{t}) +  \log q(\rvx_{1:T} | \rvx_0) \Big] 
\end{align}

\subsection{Forward process}

For the approximate posterior we assume again a Markov chain but now in the other direction as a \addidx{forward process} and with known Gaussian transitions with a variance schedule $\beta_1, \ldots, \beta_T$
\begin{align*}
    q(\rvx_{1:T} | \rvx_0) &= \prod_{t=1}^T q(\rvx_{t} | \rvx_{t-1}) \\
    q(\rvx_{t} | \rvx_{t-1}) &= \dN(\rvx_{t}; \sqrt{1-\beta_t} \rvx_{t-1}, \beta_t \mI) \enspace .
\end{align*}

\paragraph{VAE-like minimization with forward process:} 
The minimization problem \eqref{eq:VAE_loss} can now be written as
\begin{align}\label{eq:VAE_fwd}
    \Ls(\rvx_0) &= \E_{q(\rvx_{1:T} | \rvx_0)} \log \frac{\prod_{t=1}^T q(\rvx_{t} | \rvx_{t-1})}{p(\rvx_T) \prod_{t=1}^T \pt(\rvx_{t-1} | \rvx_t) } \nn
    &= \E_{q(\rvx_{1:T} | \rvx_0)} \left[ - \log p(\rvx_T) + \log \prod_{t=1}^T \frac{q(\rvx_{t} | \rvx_{t-1})}{\pt(\rvx_{t-1} | \rvx_t) } \right] \nn
    &= \E_{q(\rvx_{1:T} | \rvx_0)} \left[ - \log p(\rvx_T) + \sum_{t=1}^T \log \frac{q(\rvx_{t} | \rvx_{t-1})}{\pt(\rvx_{t-1} | \rvx_t) } \right]
\end{align}


\paragraph{Sampling from forward process:}
We can sample from forward process $q(\rvx_{1:T} | \rvx_0)$ recursively through $\rvx_t \sim \dN(\sqrt{1-\beta_t)} \rvx_{t-1}, \beta_t \mI)$ by fixing the previous value and sampling a noise $\rvepsilon \sim \dN(\mathbf{0}, \mI)$
\begin{equation}
\rvx_t = \sqrt{1-\beta_t} \rvx_{t-1} + \sqrt{\beta_t} \rvepsilon \quad 
\E(\rvx_t) = \sqrt{1-\beta_t} \rvx_{t-1} \quad
\Var(\rvx_t) = \beta_t \mI
\end{equation}
Let us indicate $\alpha_t = 1-\beta_t$ and hence $\beta_t = 1-\alpha_t$

We then get
\begin{align*}
    \rvx_1 &\sim \dN(\sqrt{\alpha_1} \rvx_0, (1 -\alpha_1) \mI) \\
    \rvx_1 &= \sqrt{\alpha_1} \rvx_0 + \sqrt{1 -\alpha_1} \rvepsilon_0 \\
\end{align*}
\begin{align*}
    \rvx_2 &\sim \dN(\sqrt{\alpha_2} \rvx_1, (1 -\alpha_2) \mI) \\
    \rvx_2 &= \sqrt{\alpha_2} \rvx_1 + \sqrt{1 -\alpha_2} \rvepsilon_1 \\
    &= \sqrt{\alpha_2}(\sqrt{\alpha_1} \rvx_0 + \sqrt{1 -\alpha_1} \rvepsilon_0) + \sqrt{1 -\alpha_2} \rvepsilon_1\\
    &= \sqrt{\alpha_1\alpha_2} \rvx_0 + \sqrt{\alpha_2 -\alpha_1\alpha_2} \rvepsilon_0 + \sqrt{1 -\alpha_2} \rvepsilon_1 \\
    \E(\rvx_2) &= \sqrt{\alpha_1\alpha_2} \rvx_0 \\
    \Var(\rvx_2) &= (\alpha_2 -\alpha_1\alpha_2 + 1 - \alpha_2) \mI = (1 - \alpha_1\alpha_2) \mI \\
    \rvx_2 &= \sqrt{\alpha_1\alpha_2} \rvx_0 + \sqrt{1 - \alpha_1\alpha_2} \rvepsilon \\
    \rvx_2 &\sim \dN(\sqrt{\alpha_1\alpha_2} \rvx_0, (1 - \alpha_1\alpha_2) \mI)
\end{align*}
\begin{align*}
    \rvx_3 &\sim \dN(\sqrt{\alpha_3} \rvx_2, (1 -\alpha_3) \mI) \\
    \rvx_2 &= \sqrt{\alpha_3} \rvx_2 + \sqrt{1 -\alpha_3} \rvepsilon_2 \\
    &= \sqrt{\alpha_3}(\sqrt{\alpha_1\alpha_2} \rvx_0 + \sqrt{\alpha_2 -\alpha_1\alpha_2} \rvepsilon_0 + \sqrt{1 -\alpha_2} \rvepsilon_1) + \sqrt{1 -\alpha_3} \rvepsilon_2\\
    &= \sqrt{\alpha_1\alpha_2\alpha_3} \rvx_0 + \sqrt{\alpha_2\alpha_3 -\alpha_1\alpha_2\alpha_3} \rvepsilon_0 + \sqrt{\alpha_3 -\alpha_2\alpha_3} \rvepsilon_1 + \sqrt{1 -\alpha_3} \rvepsilon_2\\
    \E(\rvx_2) &= \sqrt{\alpha_1\alpha_2\alpha_3} \rvx_0 \\
    \Var(\rvx_2) &= (\alpha_2\alpha_3 -\alpha_1\alpha_2\alpha_3 + \alpha_3 -\alpha_2\alpha_3 + 1 -\alpha_3) \mI = (1 - \alpha_1\alpha_2\alpha_3) \mI \\
    \rvx_2 &= \sqrt{\alpha_1\alpha_2\alpha_3} \rvx_0 + \sqrt{1 - \alpha_1\alpha_2\alpha_3} \rvepsilon \\
    \rvx_2 &\sim \dN(\sqrt{\alpha_1\alpha_2\alpha_3} \rvx_0, (1 - \alpha_1\alpha_2\alpha_3) \mI)
\end{align*}
and in general
\begin{align}\label{eq:xt_sampling}
    \rvx_t &\sim \dN(\sqrt{\alpha_t} \rvx_{t-1}, (1 -\alpha_t) \mI) \nn
    \rvx_t &= \sqrt{\alpha_t} \rvx_{t-1} + \sqrt{1 -\alpha_t} \rvepsilon_{t-1} \nn
    \rvx_t &= \sqrt{\prod_{s=1}^t \alpha_s} \rvx_0 + \sqrt{1 - \prod_{s=1}^t \alpha_s} \rvepsilon \nn
    \rvx_t &= \sqrt{\bar{\alpha}_t} \rvx_0 + \sqrt{1 - \bar{\alpha}_t} \rvepsilon \\
    \rvx_t &\sim \dN(\sqrt{\bar{\alpha}_t} \rvx_0, (1 - \bar{\alpha}_t) \mI) = q(\rvx_t | \rvx_0) \nonumber
\end{align}

In summary, instead of fixing the variance schedule $\beta_1 \ldots \beta_T$ and sampling from the forward process recursively 
\begin{equation}
    \rvx_t \sim \dN(\sqrt{1-\beta_t} \rvx_{t-1}, \beta_t \mI) = q(\rvx_t | \rvx_{t-1}) \quad \text{via} \quad \rvx_t = \sqrt{1-\beta_t} \rvx_{t-1} + \sqrt{\beta_t} \rvepsilon, \quad \rvepsilon \sim \dN(\mathbf{0}, \mI)
\end{equation}
we can fix the schedule $\bar{\alpha}_1 \ldots \bar{\alpha}_T$ and sample arbitrary timestep directly from
\begin{equation}
    \rvx_t \sim \dN(\sqrt{\bar{\alpha}_t} \rvx_0, (1 - \bar{\alpha}_t) \mI)  = q(\rvx_t | \rvx_0) \quad \text{via} \quad \rvx_t = \sqrt{\bar{\alpha}_t} \rvx_0 + \sqrt{1 - \bar{\alpha}_t} \rvepsilon, \quad \rvepsilon \sim \dN(\mathbf{0}, \mI)
\end{equation}

\paragraph{Minimization with $x_0$ conditioning:}
We can further play with the variational bound \eqref{eq:VAE_fwd}
\begin{align}
    \Ls(\rvx_0) &= \E_{q(\rvx_{1:T} | \rvx_0)} \left[ - \log p(\rvx_T) + \sum_{t=1}^T \log \frac{q(\rvx_{t} | \rvx_{t-1})}{\pt(\rvx_{t-1} | \rvx_t) } \right] \nn
    &= \E_{q(\rvx_{1:T} | \rvx_0)} \left[ - \log p(\rvx_T) + \sum_{t=2}^T \log \frac{q(\rvx_{t} | \rvx_{t-1})}{\pt(\rvx_{t-1} | \rvx_t) } + \log \frac{q(\rvx_1 | \rvx_0)}{\pt(\rvx_0 | \rvx_1) } \right] \nn
\end{align} 

Now we use the following for the forward process
\begin{align}
    q(\rvx_t, \rvx_{t-1}, \rvx_0) & = q(\rvx_t | \rvx_{t-1}, \rvx_0) q(\rvx_{t-1} | \rvx_0) q(\rvx_0) \nn
    & = q(\rvx_t | \rvx_{t-1}) q(\rvx_{t-1} | \rvx_0) q(\rvx_0) \quad \text{(Markov assumption on forward process)} \nn
    q(\rvx_t, \rvx_{t-1}, \rvx_0) & = q(\rvx_{t-1} | \rvx_t, \rvx_0) q(\rvx_t | \rvx_0) q(\rvx_0)
\end{align}
so that
\begin{align}
    q(\rvx_t | \rvx_{t-1}) &= q(\rvx_t | \rvx_{t-1}, \rvx_0) \nn
    &= \frac{q(\rvx_t, \rvx_{t-1}, \rvx_0)}{q(\rvx_{t-1} | \rvx_0) q(\rvx_0)} \nn
    &= \frac{q(\rvx_{t-1} | \rvx_t, \rvx_0) q(\rvx_t | \rvx_0) q(\rvx_0)}{q(\rvx_{t-1} | \rvx_0) q(\rvx_0)}
\end{align}
to rewrite the minimization as
\begin{align}
    \Ls(\rvx_0) &= \E_{q(\rvx_{1:T} | \rvx_0)} \left[ - \log p(\rvx_T) + \sum_{t=2}^T \log \frac{q(\rvx_{t-1} | \rvx_t, \rvx_0)}{\pt(\rvx_{t-1} | \rvx_t) } + \sum_{t=2}^T \log \frac{q(\rvx_t | \rvx_0)}{q(\rvx_{t-1} | \rvx_0)} + \log \frac{q(\rvx_1 | \rvx_0)}{\pt(\rvx_0 | \rvx_1) } \right] \nn
\end{align} 
We then observe that
\begin{align}
    \sum_{t=2}^T \log \frac{q(\rvx_t | \rvx_0)}{q(\rvx_{t-1} | \rvx_0)} &= \log \prod_{t=2}^T \frac{q(\rvx_t | \rvx_0)}{q(\rvx_{t-1} | \rvx_0)} \nn
    &= \log \frac{q(\rvx_2 | \rvx_0)}{q(\rvx_1 | \rvx_0)}\frac{q(\rvx_3 | \rvx_0)}{q(\rvx_2 | \rvx_0)}\frac{q(\rvx_4 | \rvx_0)}{q(\rvx_3 | \rvx_0)} \ldots \frac{q(\rvx_T | \rvx_0)}{q(\rvx_T-1 | \rvx_0)} \nn
    &= \log \frac{q(\rvx_T | \rvx_0)}{q(\rvx_1 | \rvx_0)}
\end{align}
and hence get for a single sample

\begin{align}\label{eq:ELBOSingleVar}
    \Ls(\rvx_0)
    &= \E_{q(\rvx_{1:T} | \rvx_0)} \left[ - \log p(\rvx_T) + \sum_{t=2}^T \log \frac{q(\rvx_{t-1} | \rvx_t, \rvx_0)}{\pt(\rvx_{t-1} | \rvx_t) } + \log \frac{q(\rvx_T | \rvx_0)}{q(\rvx_1 | \rvx_0)} + \log \frac{q(\rvx_1 | \rvx_0)}{\pt(\rvx_0 | \rvx_1) } \right] \nn
    &= \E_{q(\rvx_{T}  | \rvx_0)} \log \frac{q(\rvx_T | \rvx_0)}{p(\rvx_T)}
    - \E_{q(\rvx_1  | \rvx_0)} \log \pt(\rvx_0 | \rvx_1)
    + \E_{q(\rvx_{1:T} | \rvx_0)} \sum_{t=2}^T \log \frac{q(\rvx_{t-1} | \rvx_t, \rvx_0)}{\pt(\rvx_{t-1} | \rvx_t) } 
\end{align}


and in expectation for the complete data set
\begin{align}\label{eq:VarBoundLoss}
    \Ls &= \E_{q(\rvx_0)} \Ls(\rvx_0) \nn
    &= \E_{q(\rvx_{0:T})} \left[ - \log p(\rvx_T) + \sum_{t=2}^T \log \frac{q(\rvx_{t-1} | \rvx_t, \rvx_0)}{\pt(\rvx_{t-1} | \rvx_t) } + \log \frac{q(\rvx_T | \rvx_0)}{q(\rvx_1 | \rvx_0)} + \log \frac{q(\rvx_1 | \rvx_0)}{\pt(\rvx_0 | \rvx_1) } \right] \nn
    &= \E_{q(\rvx_{0:T})} \left[ - \log \frac{p(\rvx_T)}{q(\rvx_T | \rvx_0)} + \sum_{t=2}^T \log \frac{q(\rvx_{t-1} | \rvx_t, \rvx_0)}{\pt(\rvx_{t-1} | \rvx_t) } - \log \pt(\rvx_0 | \rvx_1)  \right] \nn
    &= \E_{q(\rvx_{0:T})} \left[ \underbrace{\KL\left(q(\rvx_T | \rvx_0) \mid\mid p(\rvx_T)\right)}_{\Ls_T} +
    \sum_{t=2}^T \underbrace{\KL\left( q(\rvx_{t-1} | \rvx_t, \rvx_0) \mid\mid \pt(\rvx_{t-1} | \rvx_t) \right)}_{\Ls_{t-1}}
    \underbrace{- \log \pt(\rvx_0 | \rvx_1)}_{\Ls_0}
     \right]
\end{align} 

\begin{notebox}[colback=red!5]
What is the missing term to complete the bound (see VAE)?
\end{notebox}

The forward process posterior $q(\rvx_{t-1} | \rvx_t, \rvx_0)$ conditioned on $\rvx_0$ is tractable and can be compared to the learned reversed process $\pt(\rvx_{t-1} | \rvx_t)$
\begin{equation}
    q(\rvx_{t-1} | \rvx_t, \rvx_0) = \frac{q(\rvx_t | \rvx_{t-1}) q(\rvx_{t-1} | \rvx_0) }{q(\rvx_t | \rvx_0) }
\end{equation}

Pdf of exponential family distributions can be represented in a form
\begin{equation}
    p(\rvx; \veta) = h(\rvx) \exp \left( \veta^\Ts \mT(\rvx) - \mA(\veta) \right) \enspace .
\end{equation}

For multivariate Guassian distribution we get \cite{escudero_multivariate_2020}
\begin{align*}
    \dN(\rvx ; \vmu, \mSigma)
    &= (2 \pi)^{-k / 2} \det(\mSigma)^{-1 / 2} \exp \left(-\frac{1}{2}(\rvx-\vmu)^\Ts \mSigma^{-1}(\rvx - \vmu)\right) \\
    &= \exp \left(-\frac{1}{2}\rvx^\Ts \mSigma^{-1} \rvx + \rvx^\Ts \mSigma^{-1} \vmu - \frac{1}{2}\vmu^\Ts \mSigma^{-1} \vmu - \frac{k}{2} \log (2 \pi) - \frac{1}{2} \log \det(\mSigma) \right) \\
    &= \exp \left(-\frac{1}{2} \Tr(\mSigma^{-1} \rvx \rvx^\Ts ) + \Tr( \mSigma^{-1} \vmu \rvx^\Ts ) - \frac{1}{2}\vmu^\Ts \mSigma^{-1} \vmu - \frac{k}{2} \log (2 \pi) - \frac{1}{2} \log \det(\mSigma) \right) \\
    &= \exp \left(-\frac{1}{2} \vvec(\mSigma^{-1})^\Ts \vvec(\rvx \rvx^\Ts ) + (\mSigma^{-1} \vmu)^\Ts \rvx - \frac{1}{2}\vmu^\Ts \mSigma^{-1} \vmu - \frac{k}{2} \log (2 \pi) - \frac{1}{2} \log \det(\mSigma) \right) 
\end{align*}
From which we have 
\begin{align*}
    \veta &= \begin{bmatrix}
        \mSigma^{-1} \vmu \\
        -\frac{1}{2} \vvec(\mSigma^{-1})
    \end{bmatrix} \\
    T(\rvx) &= \begin{bmatrix}
        \rvx \\
        \vvec(\rvx \rvx^\Ts )
    \end{bmatrix} \\
    \mA(\veta) &= - \frac{1}{2} \left( \vmu^\Ts \mSigma^{-1} \vmu + k \log (2 \pi) + \log \det(\mSigma)\right)
\end{align*}

For the forward process we have
\begin{align*}
    q(\rvx_{t} | \rvx_{t-1}) &= \dN(\rvx_{t}; \sqrt{1-\beta_t} \rvx_{t-1}, \beta_t \mI) \\
    &= \exp \left(-\frac{1}{2\beta_t} \rvx_t^\Ts \rvx_t + \frac{\sqrt{\alpha_t}}{\beta_t} \rvx_t^\Ts \rvx_{t-1} - \frac{\alpha_t}{2\beta_t}\rvx_{t-1}^\Ts \rvx_{t-1} - \frac{k}{2} \log (2 \pi) - \frac{1}{2} \log k \beta_t \right) 
\end{align*} 

\begin{align*}
    q(\rvx_t | \rvx_0) &= \dN(\rvx_t; \sqrt{\bar{\alpha}_t} \rvx_0, (1 - \bar{\alpha}_t) \mI) \\
    &= \exp \left(-\frac{1}{2(1 - \bar{\alpha}_t)} \rvx_t^\Ts \rvx_t + \frac{\sqrt{\bar{\alpha}_t}}{(1 - \bar{\alpha}_t)} \rvx_t^\Ts \rvx_{0} - \frac{\bar{\alpha}_t}{2(1 - \bar{\alpha}_t)}\rvx_{0}^\Ts \rvx_{0} - \frac{k}{2} \log (2 \pi) - \frac{1}{2} \log k (1 - \bar{\alpha}_t) \right) 
\end{align*}

\begin{align*}
    q(\rvx_{t-1} | \rvx_0) &= \dN(\rvx_t; \sqrt{\bar{\alpha}_{t-1}} \rvx_0, (1 - \bar{\alpha}_{t-1}) \mI) \\
    &= \exp \left(-\frac{1}{2(1 - \bar{\alpha}_{t-1})} \rvx_{t-1}^\Ts \rvx_{t-1} + \frac{\sqrt{\bar{\alpha}_{t-1}}}{(1 - \bar{\alpha}_{t-1})} \rvx_{t-1}^\Ts \rvx_{0} - \frac{\bar{\alpha}_{t-1}}{2(1 - \bar{\alpha}_{t-1})}\rvx_{0}^\Ts \rvx_{0} - \frac{k}{2} \log (2 \pi) - \frac{1}{2} \log k (1 - \bar{\alpha}_{t-1}) \right) 
\end{align*}

And hence
\begin{align*}
    q(\rvx_{t-1} | \rvx_t, \rvx_0) &= \frac{q(\rvx_t | \rvx_{t-1}) q(\rvx_{t-1} | \rvx_0) }{q(\rvx_t | \rvx_0) } \\
    &= \exp \left(
        \frac{\sqrt{\alpha_t}}{\beta_t} \rvx_t^\Ts \rvx_{t-1} - \frac{\alpha_t}{2\beta_t}\rvx_{t-1}^\Ts \rvx_{t-1}-\frac{1}{2(1 - \bar{\alpha}_{t-1})} \rvx_{t-1}^\Ts \rvx_{t-1} + \frac{\sqrt{\bar{\alpha}_{t-1}}}{(1 - \bar{\alpha}_{t-1})} \rvx_{t-1}^\Ts \rvx_{0}
        - \mA(\veta) \right) \\
        &= \exp \left( -\frac{1}{2}\left( \frac{\alpha_t}{\beta_t} + \frac{1}{(1 - \bar{\alpha}_{t-1})} \right)\rvx_{t-1}^\Ts \rvx_{t-1}
        + \rvx_{t-1}^\Ts \left( \frac{\sqrt{\alpha_t}}{\beta_t}\rvx_t + \frac{\sqrt{\bar{\alpha}_{t-1}}}{(1 - \bar{\alpha}_{t-1})} \rvx_0 \right)
        - \mA(\veta) 
        \right)  \enspace .
\end{align*}
where $\mA(\veta)$ is the log-partition function (cummulant) contains all the normalizing terms not depending on $\rvx_{t-1}$.

From this we have that the covariance of the distribution $q(\rvx_{t-1} | \rvx_t, \rvx_0)$ is 
\begin{align*}
    \mSigma &= \left(\frac{\alpha_t}{\beta_t} + \frac{1}{(1 - \bar{\alpha}_{t-1})} \right)^{-1} \mI
    = \left(\frac{\beta_t + \alpha_t - \bar{\alpha}_t}{\beta_t(1 - \bar{\alpha}_{t-1})} \right)^{-1} \mI
    = \frac{\beta_t(1 - \bar{\alpha}_{t-1})}{1 - \bar{\alpha}_t} \mI = \bar{\beta}_t \mI
    \enspace ,
\end{align*}
and the mean is
\begin{align*}
    \vmu(\rvx_t, \rvx_0) &= \left( \frac{\sqrt{\alpha_t}}{\beta_t}\rvx_t + \frac{\sqrt{\bar{\alpha}_{t-1}}}{(1 - \bar{\alpha}_{t-1})} \rvx_0 \right) \frac{\beta_t(1 - \bar{\alpha}_{t-1})}{1 - \bar{\alpha}_t} \\
    &= \frac{\sqrt{\alpha_t}(1 - \bar{\alpha}_{t-1})}{1 - \bar{\alpha}_t}\rvx_t
    + \frac{(1-\alpha_t)\sqrt{\bar{\alpha}_{t-1}}}{(1 - \bar{\alpha}_t)} \rvx_0
\end{align*}
So that $q(\rvx_{t-1} | \rvx_t, \rvx_0) = \dN(\rvx_{t-1}; \vmu(\rvx_t, \rvx_0), \bar{\beta}_t \mI)$.

From \eqref{eq:xt_sampling} we know that we can sample $\rvx_t$ in the forward diffusion directly from $\rvx_0$ as $\rvx_t = \sqrt{\bar{\alpha}_t} \rvx_0 + \sqrt{1 - \bar{\alpha}_t} \rvepsilon_t$ with $\rvepsilon_t \sim \dN(\mathbf{0}, \mI)$. 
Hence we can also recover the $\rvx_0$ from the sample (if we know the noise) as 
$\rvx_0 = \frac{1}{\sqrt{\bar{\alpha}_t}} \left( \rvx_t - \sqrt{1 - \bar{\alpha}_t} \rvepsilon_t \right)$.
This also means that the posterior is
\begin{equation}\label{eq:x0_posterior}
    q(\rvx_0 | \rvx_t) = \dN(\rvx_0 ; \frac{1}{\sqrt{\bar{\alpha}_t}}\rvx_t , \frac{1-\bar{\alpha}_t}{\bar{\alpha}_t}\mI)
\end{equation}

For the mean of the distribution we thus get in terms of the known error $\rvepsilon_t$
\begin{align}\label{eq:forwardPosteriorEps}
    \vmu(\rvx_t, \rvepsilon_t) &= \frac{\sqrt{\alpha_t}(1 - \bar{\alpha}_{t-1})}{1 - \bar{\alpha}_t}\rvx_t
    + \frac{(1-\alpha_t)\sqrt{\bar{\alpha}_{t-1}}}{(1 - \bar{\alpha}_t)} \rvx_0 \nn
    &= \frac{\sqrt{\alpha_t}(1 - \bar{\alpha}_{t-1})}{1 - \bar{\alpha}_t}\rvx_t
    + \frac{(1-\alpha_t)\sqrt{\bar{\alpha}_{t-1}}}{(1 - \bar{\alpha}_t)}
    \frac{1}{\sqrt{\bar{\alpha}_t}} \left( \rvx_t - \sqrt{1 - \bar{\alpha}_t} \rvepsilon_t  \right) \nn
    &= \frac{\alpha_t(1 - \bar{\alpha}_{t-1})+ (1 - \alpha_t)}{(1 - \bar{\alpha}_t)\sqrt{\alpha_t}}\rvx_t
    - \frac{(1-\alpha_t)}{\sqrt{1 - \bar{\alpha}_t}\sqrt{\alpha_t}}
    \rvepsilon_t \nn
    &= \frac{1}{\sqrt{\alpha_t}} \left( 
        \rvx_t - \frac{(1-\alpha_t)}{\sqrt{1 - \bar{\alpha}_t}} \rvepsilon_t
    \right)
\end{align}
So that the posterior of the forward process can be conditionend on the known error sample $q(\rvx_{t-1} | \rvx_t, \rvepsilon_t) = \dN(\rvx_{t-1}; \vmu(\rvx_t, \rvepsilon_t), \bar{\beta}_t \mI)$.

\section{Simplifying the loss function - denoising autoencoder}

The $\Ls_T$ term in \eqref{eq:VarBoundLoss} has no learnable parameters. 
The prior $p(\rvx_T)$ is standard normal and $q(\rvx_T | \rvx_0)$ parameters depend only on the forward variance schedule through \eqref{eq:xt_sampling}.
It can just be dropped from the objective.

General the KL divergence for two multivariate Gaussians is as follows
\begin{align*}
    \KL(\dN(\vmu_q, \mSigma_q) \mid\mid \dN(\vmu_p, \mSigma_p)) = 
    \frac{1}{2} \left( \log \frac{\det(\mSigma_p)}{\det(\mSigma_q)} - k + \Tr \, (\mSigma_p^{-1} \mSigma_q) + (\vmu_q - \vmu_p)^\Ts \mSigma_p^{-1} (\vmu_q - \vmu_p) \right)
\end{align*}

For the $\Ls_{t-1}$ terms the posterior of the forward process $q(\rvx_{t-1} | \rvx_t, \rvx_0) = \dN(\rvx_{t-1}; \vmu(\rvx_t, \rvx_0), \bar{\beta}_t \mI)$ has no learnable parameters.
We fix the variance in the reverse process $\pt(\rvx_{t-1} | \rvx_t) = \dN(\rvx_{t-1}; \mt(\rvx_t, t), \st(\rvx_t, t))$ so that $\st(\rvx_t, t) = \sigma^2_t \mI$, where $\sigma^2_t$ is a known (not learnable) function of the forward variance schedule.
The minimization of the KL divergence hence simplifies to:
\begin{gather}\label{eq:Lt-1norm}
    \amint \E_{q(\rvx_0, \rvx_t)} \KL\left( q(\rvx_{t-1} | \rvx_t, \rvx_0) \mid\mid \pt(\rvx_{t-1} | \rvx_t) \right) \nn
    \text{is equivalent to} \nn
    \amint \E_{q(\rvx_0, \rvx_t)} \frac{1}{2 \sigma_t^2} \norm{\vmu(\rvx_t, \rvx_0) - \mt(\rvx_t, t)}_2^2
\end{gather}
and hence we can train $\mt$ to approximate the mean of the forward process posterior.

However, we know from \eqref{eq:forwardPosteriorEps} that mean of the forward posterior can be written with respect to $\rvx_t$ and the noise which $\rvepsilon_t$ which was used to generate if from $\rvx_0$.
\begin{equation*}
    \vmu(\rvx_t, \rvepsilon_t) = \frac{1}{\sqrt{\alpha_t}} \left( 
        \rvx_t - \frac{(1-\alpha_t)}{\sqrt{1 - \bar{\alpha}_t}} \rvepsilon_t  \right)  
\end{equation*}
We can therefore choose the parameterization for the mean of the reverse process as
\begin{equation*}
    \mt(\rvx_t, t) = \frac{1}{\sqrt{\alpha_t}} 
    \left( \rvx_t - \frac{(1-\alpha_t)}{\sqrt{1 - \bar{\alpha}_t}} \epst(\rvx_t, t)  \right)  
\end{equation*}

In this parametrization is the reverse process
\begin{equation*}
    \pt(\rvx_{t-1} | \rvx_t) = \dN(\rvx_{t-1}; \frac{1}{\sqrt{\alpha_t}} 
    \left( \rvx_t - \frac{(1-\alpha_t)}{\sqrt{1 - \bar{\alpha}_t}} \epst(\rvx_t, t) \right), \sigma_t^2 \mI)
\end{equation*}
and we can sample $\rvx_{t-1}$ as
\begin{equation*}
    \rvx_{t-1} = \left( \rvx_t - \frac{(1-\alpha_t)}{\sqrt{1 - \bar{\alpha}_t}} \epst(\rvx_t, t) \right) + \sigma_t \rvz, \qquad \rvz \sim \dN(\mathbf{0}, \mI)
\end{equation*}

Plugging these back to the optimisation from \eqref{eq:Lt-1norm} we get
\begin{align*}
    \norm{\vmu(\rvx_t, \rvx_0) - \mt(\rvx_t, t)}_2^2 &=
    \norm{\frac{1}{\sqrt{\alpha_t}} \left( \rvx_t - \frac{(1-\alpha_t)}{\sqrt{1 - \bar{\alpha}_t}} \rvepsilon_t  \right) 
    - \frac{1}{\sqrt{\alpha_t}} \left( \rvx_t - \frac{(1-\alpha_t)}{\sqrt{1 - \bar{\alpha}_t}} \epst(\rvx_t, t)  \right)
    }_2^2 \nn
    &= \norm{
    \frac{(1-\alpha_t)}{\sqrt{\alpha_t}\sqrt{1 - \bar{\alpha}_t}} \left( \rvepsilon_t - \epst(\rvx_t, t) \right)
    }_2^2 \nn
    &= \frac{(1-\alpha_t)^2}{\alpha_t(1 - \bar{\alpha}_t)}
    \norm{\rvepsilon_t - \epst( \sqrt{\bar{\alpha}_t} \rvx_0 + \sqrt{1 - \bar{\alpha}_t} \rvepsilon_t, t) }_2^2
\end{align*}

This means that the $\Ls_{t-1}$ terms of the objective boil down to 
\begin{equation*}
    \Ls_{t-1}: \, \amint \E_{q(\rvx_0)\dN(\rvepsilon_t; \mathbf{0}, \mI)} \,
    \frac{(1-\alpha_t)^2}{2\sigma_t^2 \alpha_t(1 - \bar{\alpha}_t)} \norm{\rvepsilon_t - \epst( \underbrace{\sqrt{\bar{\alpha}_t} \rvx_0 + \sqrt{1 - \bar{\alpha}_t} \rvepsilon_t}_{\rvx_t}, t) }_2^2 \enspace ,
\end{equation*}
whereby the diffusion model is trained to predict the noise from the noised image and the corresponding timestamp.

In \cite{ho_denoising_2020} they model $\pt(\rvx_0 | \rvx_1)$ as independent discrete decoder over the image pixels. Check the paper for details.

They also found that the don't need to re-weight the loss terms so that in the end we have a simple objective
\begin{equation*}
    \Ls_{simple}: \amint \, \sum_{t=1}^T \norm{\rvepsilon_t - \epst( \underbrace{\sqrt{\bar{\alpha}_t} \rvx_0 + \sqrt{1 - \bar{\alpha}_t} \rvepsilon_t}_{\rvx_t}, t) }_2^2 \enspace ,
\end{equation*}
where the $\pt(\rvx_0 | \rvx_1)$ has been subsumed into the loss.
\begin{notebox}[colback=red!5]
    I don't quite see, how this happens but seems not very critical.
\end{notebox}

The final point here is that we can use the trained model $\epst$ to predict the original image $\widehat{\rvx}_0$ from the noised image $\rvx_t$ and the timestamp as
\begin{equation}
    \rvx_0 \approx \widehat{\rvx}_0 = \frac{1}{\sqrt{\bar{\alpha}_t}} \left( \rvx_t - \sqrt{1 - \bar{\alpha}_t} \epst (\rvx_t, t) \right) = \ut(\rvx_t, t)
\end{equation}
and we call this function $\ut(\rvx_t, t)$.

\paragraph{Some comments on this}
It is important to understand the properties of this approximator.
When $\rvx_t$ is the result of the forward process, the mean of this is
\begin{align*}
    \E(\widehat{\rvx}_0) &= \E \left[ \frac{1}{\sqrt{\bar{\alpha}_t}} \left( \sqrt{\bar{\alpha}_t} \rvx_0 + \sqrt{1 - \bar{\alpha}_t} \rvepsilon_t - \sqrt{1 - \bar{\alpha}_t} \epst (\sqrt{\bar{\alpha}_t} \rvx_0 + \sqrt{1 - \bar{\alpha}_t} \rvepsilon_t, t) \right) \right] \nn
    &= \rvx_0 - \frac{\sqrt{1 - \bar{\alpha}_t}}{\sqrt{\bar{\alpha}_t}} \E(\epst )
\end{align*}
Hence the bias is a function of the bias of the predictor $\epst$.

The variance 
\begin{align*}
    \Var(\widehat{\rvx}_0) &= \Var \left[ \frac{1}{\sqrt{\bar{\alpha}_t}} \left( \sqrt{\bar{\alpha}_t} \rvx_0 + \sqrt{1 - \bar{\alpha}_t} \rvepsilon_t - \sqrt{1 - \bar{\alpha}_t} \epst (\sqrt{\bar{\alpha}_t} \rvx_0 + \sqrt{1 - \bar{\alpha}_t} \rvepsilon_t, t) \right) \right] \nn
    &= \frac{1 - \bar{\alpha}_t}{\bar{\alpha}_t} (\mI + \Var(\epst)) \enspace .
\end{align*}
It can be expected that the variance $\Var(\epst)$ is larger for bigger $t$ (further away from the original image).
The variance schedule has an effect on this through $\bar{\alpha}_t$.
\begin{notebox}[colback=red!5]
    I can't think it through but probably worse exploring a bit more.
\end{notebox}

\begin{notebox}[colback=red!5]
    When $\rvx_t$ comes from the reverse process the moments are not the same. Again, I cannot think it through now.
\end{notebox}


\clearpage

\section{Basics of classifier guidance}\label{sec:classifier_guidance}

\begin{notebox}
    \textbf{Using:} 
    \fullcite{dhariwal_diffusion_2021}
\end{notebox}

\subsection{Classifier trained on noisy images}

Let's assume a classifier $\pf(y | \rvx_t)$ trained on the noisy images $\rvx_t$ and use the gradients $\gradxt \pf(y | \rvx_t)$ to guide the diffusion sampling. 

% !TEX root = main.tex
\clearpage

\section{Denoising diffusion implicit models}\label{sec:ddim}

\begin{notebox}
    \fullcite{song_denoising_2021}
\end{notebox}

\section{Background DDPM}\label{sec:ddim_ddpm}

We start the same as in the DDPM of \cite{ho_denoising_2020}.

Goal is to learn model $\pt(\rvx_0) \approx q(\rvx_0)$ approximating the true data distribution.
We formulate the model as latent variable with latents $\rvx_{1:T}$

\begin{align}
    \pt(\rvx_0) 
    = \int \pt(\rvx_{0:T})\, d\rvx_{1:T} 
    = \int \pt(\rvx_{1:T})\,\pt(\rvx_0 \mid \rvx_{1:T}) \, d\rvx_{1:T} 
    = \int \pt(\rvx_T) \prod_{t=1}^T \pt(\rvx_{t-1} \mid \rvx_{t})\, d\rvx_{1:T} 
    \enspace .
\end{align}

This is the diffusion \textit{reverse or generative process}. 

The posterior of the latents is
\begin{equation}
    \pt(\rvx_{1:T} \mid \rvx_0)
    = \frac{\pt(\rvx_{1:T})\,\pt(\rvx_0 \mid \rvx_{1:T})}{\pt(\rvx_0)}
    = \frac{\pt(\rvx_{0:T})}{\pt(\rvx_0)} \enspace .
\end{equation}
We approximate the posterior by a fixed \textit{encoder, forward process or inference distribution} $q(\rvx_{1:T} \mid \rvx_0) \approx \pt(\rvx_{1:T} \mid \rvx_0)$.

We learn the model parameters $\theta$ by maximizing the ELBO
\begin{align}
    \E_{q(\rvx_0)} \left[ \log \pt(\rvx_0) \right] & = \E_{q(\rvx_0)} \left[\log \int \pt(\rvx_{0:T})\, d\rvx_{1:T} \right] \nn
    & = \E_{q(\rvx_0)} \left[ \log \int q(\rvx_{1:T} \mid \rvx_0) \frac{\pt(\rvx_{0:T})}{q(\rvx_{1:T} \mid \rvx_0)} d\rvx_{1:T} \right] \nn
    & \geq \E_{q(\rvx_0)} \left[ \int q(\rvx_{1:T} \mid \rvx_0) \log  \frac{\pt(\rvx_{0:T})}{q(\rvx_{1:T} \mid \rvx_0)} d\rvx_{1:T} \right] \nn
    & = \E_{q(\rvx_0)} \E_{q(\rvx_{1:T} \mid \rvx_0)} \log  \frac{\pt(\rvx_{0:T})}{q(\rvx_{1:T} \mid \rvx_0)} \nn
    & = \E_{q(\rvx_{0:T})} \log  \frac{\pt(\rvx_{0:T})}{q(\rvx_{1:T} \mid \rvx_0)} = \text{ELBO} \enspace ,
\end{align}
which is obviously equivalent to minimizing the negative ELBO
\begin{equation}
    \argmin_{\theta} - \text{ELBO} = \argmin_{\theta} \E_{q(\rvx_{0:T})} - \log \frac{\pt(\rvx_{0:T})}{q(\rvx_{1:T} \mid \rvx_0)} \enspace .
\end{equation}

In DDPM the forward process was fixed as a Markov chain, such that
\begin{equation}
    q(\rvx_{0:T}) = q(\rvx_0) \prod_{t=1}^T q(\rvx_t \mid \rvx_{t-1}), 
    \quad
    q(\rvx_t \mid \rvx_{t-1}) \sim \dN\left(\rvx_t;\, 
    \sqrt{\frac{\alpha_t}{\alpha_{t-1}}}
    \rvx_{t-1}, \left(1 - \frac{\alpha_t}{\alpha_{t-1}}\right) \mI\right) \enspace ,
\end{equation}
with the following link to the initial notation of \cite{ho_denoising_2020}
\begin{equation}
    \alpha_t = \prod_{s=1}^t (1 - \beta_t), \quad \left(1 - \frac{\alpha_t}{\alpha_{t-1}}\right) = \beta_t, \quad \sqrt{\frac{\alpha_t}{\alpha_{t-1}}} = \sqrt{1-\beta_t} \enspace .
\end{equation}

By the same logic as in \cite{ho_denoising_2020} it also holds that
\begin{equation}
    q(\rvx_t \mid \rvx_0) = \int q(\rvx_{1:t} \mid \rvx_0) d\rvx_{1:(t-1)} 
    = \dN \left( \rvx_t;
    \sqrt{\alpha_t} \rvx_0,
    (1 - \alpha_t) \mI
    \right) \enspace , 
\end{equation}
with $\lim_{t \to \infty} \alpha_t = 0$ and hence $\lim_{t \to \infty} q(\rvx_t \mid \rvx_0) = \dN(\mathbf{0}, \mI)$.

Observe that by the Markov assumption on the forward process we have
\begin{equation}
    q(\rvx_t \mid \rvx_{t-1}) = q(\rvx_t \mid \rvx_{t-1}, \rvx_0)
    = \frac{q(\rvx_t, \rvx_{t-1}, \rvx_0)}{q(\rvx_{t-1} \mid \rvx_0) q(\rvx_0)}
    = \frac{q(\rvx_{t-1} \mid \rvx_{t}, \rvx_0) q(\rvx_{t} \mid \rvx_0) q(\rvx_0)}{q(\rvx_{t-1} \mid \rvx_0) q(\rvx_0)} \enspace .
\end{equation}
We can use it in the ELBO
\begin{align}
    \text{ELBO} & = \E_{q(\rvx_{0:T})} \log  \frac{\pt(\rvx_{0:T})}{q(\rvx_{1:T} \mid \rvx_0)} \nn
    & = \E_{q(\rvx_{0:T})} \left[ \log \pt(\rvx_T) + \log \prod_{t=1}^T \frac{\pt(\rvx_{t-1} \mid \rvx_{t})}{q(\rvx_t \mid \rvx_{t-1})} \right] \nn
    & = \E_{q(\rvx_{0:T})} \left[ \log \pt(\rvx_T) + \log \prod_{t=1}^T 
    \frac{\pt(\rvx_{t-1} \mid \rvx_{t}) q(\rvx_{t-1} \mid \rvx_0)}
    {q(\rvx_{t-1} \mid \rvx_{t}, \rvx_0) q(\rvx_{t} \mid \rvx_0)} \right] \nn
    & = \E_{q(\rvx_{0:T})} \left[ \log \pt(\rvx_T) + \log \prod_{t=2}^T 
    \frac{\pt(\rvx_{t-1} \mid \rvx_{t})}
    {q(\rvx_{t-1} \mid \rvx_{t}, \rvx_0)} 
    + \log \frac{\pt(\rvx_{0} \mid \rvx_{1})}{q(\rvx_{T} \mid \rvx_0)}
    \right] \nn
    & = \E_{q(\rvx_{0:T})} \left[ \log \pt(\rvx_{0} \mid \rvx_{1}) + \log \prod_{t=2}^T 
    \frac{\pt(\rvx_{t-1} \mid \rvx_{t})}
    {q(\rvx_{t-1} \mid \rvx_{t}, \rvx_0)}
    \right] \nn
    & = \E_{q(\rvx_{0:T})} \left[ \log \pt(\rvx_{0} \mid \rvx_{1}) 
    - \sum_{t=2}^T \log  \frac{q(\rvx_{t-1} \mid \rvx_{t}, \rvx_0)}
    {\pt(\rvx_{t-1} \mid \rvx_{t})}
    \right] \nn
    & = \E_{q(\rvx_{0:1})} \log \pt(\rvx_{0} \mid \rvx_{1}) 
    - \E_{q(\rvx_{0})} \sum_{t=2}^T \KL\left( q(\rvx_{t-1} \mid \rvx_{t}, \rvx_0) \mid\mid \pt(\rvx_{t-1} \mid \rvx_{t}) \right)
    \enspace ,
\end{align}
where we assume that $\pt(\rvx_T) = q(\rvx_{T} \mid \rvx_0) = \dN(\mathbf{0}, \mI)$ and therefore drop it (also we cannot influence these by training so can be).

We further have from \cite{ho_denoising_2020}
\begin{equation}
    q(\rvx_{t-1} | \rvx_t, \rvx_0) = \dN(\rvx_{t-1};\, \vmu(\rvx_t, \rvx_0), \bar{\beta}_t \mI) \enspace ,
\end{equation}
where
\begin{equation}
    \vmu(\rvx_t, \rvx_0) = \frac{\beta_t\sqrt{\alpha_{t-1}}}{(1 - \alpha_t)} \rvx_0
    + \frac{\sqrt{1-\beta_t}(1 - \alpha_{t-1})}{1 - \alpha_t}\rvx_t
\end{equation}
and 
\begin{equation}
    \bar{\beta}_t = \frac{\beta_t(1 - \alpha_{t-1})}{1 - \alpha_t} 
\end{equation}

When we put $\pt(\rvx_{t-1} \mid \rvx_{t}) = \dN \left(\rvx_{t-1};\, \vmu_{\theta}(\rvx_t,t), \sigma_t^2 \mI \right)$ we get for the KL divergences
\begin{equation}
    \KL\left( q(\rvx_{t-1} \mid \rvx_{t}, \rvx_0) \mid\mid \pt(\rvx_{t-1} \mid \rvx_{t}) \right) 
    = \frac{1}{2 \sigma_t^2} \lVert \vmu(\rvx_t, \rvx_0) -  \vmu_{\theta}(\rvx_t,t)\rVert_{2}^{2}
\end{equation}

Using the fact that $q(\rvx_t \mid \rvx_0)
= \dN \left( \rvx_t;
\sqrt{\alpha_t} \rvx_0,
(1 - \alpha_t) \mI
\right)
$, we can sample $\rvx_t$ as
\begin{equation}
    \rvx_t = \sqrt{\alpha_t} \rvx_0 + \sqrt{(1 - \alpha_t)} \rvepsilon, \quad \rvepsilon \sim \dN(\vzero, \mI)
\end{equation}
and hence after we have sampled $\rvepsilon$ we can recover $\rvx_0$ from $\rvx_t$ as
\begin{equation}
    \rvx_0 = \frac{\rvx_t - \sqrt{(1 - \alpha_t)} \rvepsilon}{\sqrt{\alpha_t}} \enspace .
\end{equation}
With this we can 
\begin{align}
    \vmu(\rvx_t, \rvx_0) & = \frac{\beta_t\sqrt{\alpha_{t-1}}}{(1 - \alpha_t)} \rvx_0
    + \frac{\sqrt{1-\beta_t}(1 - \alpha_{t-1})}{1 - \alpha_t}\rvx_t \nn
    & = \frac{\beta_t\sqrt{\alpha_{t-1}}}{(1 - \alpha_t)\sqrt{\alpha_t}} 
    \left(\rvx_t - \sqrt{(1 - \alpha_t)} \rvepsilon \right)
    + \frac{\sqrt{1-\beta_t}(1 - \alpha_{t-1})}{1 - \alpha_t}\rvx_t \nn
    & = \frac{\beta_t}{(1 - \alpha_t)\sqrt{1 - \beta_t}} \left(\rvx_t - \sqrt{(1 - \alpha_t)} \rvepsilon \right)
    + \frac{\sqrt{1-\beta_t}(1 - \alpha_{t-1})}{1 - \alpha_t}\rvx_t \nn
    & = \frac{\beta_t + 1-\beta_t - \alpha_{t}}{(1 - \alpha_t)\sqrt{(1 - \beta_t)}} \rvx_t
    - \frac{\beta_t}{\sqrt{1 - \beta_t}\sqrt{(1 - \alpha_t)}} \rvepsilon \nn
    & = \frac{1}{\sqrt{1 - \beta_t}} \left(\rvx_t - \frac{\beta_t}{\sqrt{(1 - \alpha_t)}} \rvepsilon \right)
    = \vmu(\rvx_t, \rvepsilon) \enspace .
\end{align}
We can set 
\begin{equation}
    \vmu_{\theta}(\rvx_t,t) = \frac{1}{\sqrt{1 - \beta_t}} \left(\rvx_t - \frac{\beta_t}{\sqrt{(1 - \alpha_t)}} \rvepsilon_\theta(\rvx_t, t) \right)
\end{equation}
and therefore get
\begin{align}
    \frac{1}{2 \sigma_t^2} \lVert \vmu(\rvx_t, \rvx_0) -  \vmu_{\theta}(\rvx_t,t)\rVert_{2}^{2} 
    & =  \frac{1}{2 \sigma_t^2} \left\lVert 
        \frac{1}{\sqrt{1 - \beta_t}} \left(\rvx_t - \frac{\beta_t}{\sqrt{(1 - \alpha_t)}} \rvepsilon \right)
    -  \frac{1}{\sqrt{1 - \beta_t}} \left(\rvx_t - \frac{\beta_t}{\sqrt{(1 - \alpha_t)}} \rvepsilon_\theta(\rvx_t, t) \right) \right\rVert_{2}^{2} \nn
    & = \frac{\beta_t^2}{2 \sigma_t^2 \, (1-\beta_t) (1-\alpha_t)}
    \left\lVert 
        \rvepsilon - \rvepsilon_{\theta}(\sqrt{\alpha_t} \rvx_0 + \sqrt{(1 - \alpha_t)} \rvepsilon, t)
    \right\rVert_{2}^{2} \enspace .
\end{align}

More generally, we can write the loss as
\begin{equation*}
    \sum_{t=2}^T \E_{q(\rvx_0)} 
    \KL\left( q(\rvx_{t-1} \mid \rvx_{t}, \rvx_0) \mid\mid \pt(\rvx_{t-1} \mid \rvx_{t}) \right) 
    = \sum_{t=2}^T \E_{q(\rvx_0)} \gamma_t \left\lVert 
        \rvepsilon - \rvepsilon_{\theta}(\sqrt{\alpha_t} \rvx_0 + \sqrt{(1 - \alpha_t)} \rvepsilon, t)
    \right\rVert_{2}^{2}
\end{equation*}
\begin{equation}
    \L_{\gamma}(\rvepsilon_{\theta}) = \sum_{t=2}^T \gamma_t \E_{q(\rvx_0), \rvepsilon_t}
    \left\lVert 
        \rvepsilon_t - \rvepsilon_{\theta}^{t}(\sqrt{\alpha_t} \rvx_0 + \sqrt{(1 - \alpha_t)} \rvepsilon_t)
    \right\rVert_{2}^{2}
\end{equation}
\begin{notebox}[colback=red!5]
    In the paper they start the sum from $t=1$. I still do not quite understand the first step.
\end{notebox}

\section{Moving onto DDIM}\label{sec:ddim_ddim}
We take a different assumption for $q(\rvx_{1:T} \mid \rvx_0)$ formulated as \textit{reverse process} 
\begin{equation}
    q(\rvx_{1:T} \mid \rvx_0) = q_\sigma(\rvx_T | \rvx_0) \prod_{t=2}^T q_\sigma(\rvx_{t-1} \mid \rvx_t, \rvx_0) \enspace ,
\end{equation}
where
\begin{equation}
    q_\sigma(\rvx_{t-1} \mid \rvx_t, \rvx_0) 
    = \dN \left( \rvx_{t-1};
    \sqrt{\alpha_{t-1}}\rvx_0
    + \sqrt{1 - \alpha_{t-1} - \sigma_t^2} \, \frac{\rvx_t - \sqrt{\alpha_t}\rvx_0}{\sqrt{1-\alpha_t}}, \, \sigma_t^2 \mI
    \right)
\end{equation}
and
\begin{equation}
    q_\sigma(\rvx_T | \rvx_0) = \dN \left(  \rvx_T;
    \sqrt{\alpha_T} \rvx_0, (1-\alpha_T) \mI \enspace .
    \right)
\end{equation}

This is chosen so that it still holds (as in the DDPM) that
\begin{equation}
    q_{\sigma}(\rvx_t \mid \rvx_0) = \int q(\rvx_{1:t} \mid \rvx_0) d\rvx_{1:(t-1)} 
    = \dN \left( \rvx_t;
    \sqrt{\alpha_t} \rvx_0,
    (1 - \alpha_t) \mI
    \right) \enspace .
\end{equation}
This hold by assumption for $q_{\sigma}(\rvx_T \mid \rvx_0)$. We can start from $t=T$ and then prove by induction that it holds for all $t$.
We use marginalization formula
\begin{equation}
    q_{\sigma}(\rvx_{t-1} \mid \rvx_0) = \int q_\sigma(\rvx_{t-1} \mid \rvx_t, \rvx_0) \, q_{\sigma}(\rvx_{t} \mid \rvx_0) d \rvx_t \enspace .
\end{equation}
The $q_{\sigma}$ on the right side are both gaussians and Bisshop 2.115 says that
\begin{align}
    q_{\sigma}(\rvx_{t-1} \mid \rvx_0) & = \dN \left( \rvx_{t-1};\, \sqrt{\alpha_{t-1}}\rvx_0
    + \sqrt{1 - \alpha_{t-1} - \sigma_t^2} \, \frac{\sqrt{\alpha_t} \rvx_0 - \sqrt{\alpha_t}\rvx_0}{\sqrt{1-\alpha_t}}, \sigma_t^2 \mI  + \frac{1 - \alpha_{t-1} - \sigma_t^2}{1-\alpha_t}(1 - \alpha_t) \mI
    \right) \nn
    & = \dN \left( \rvx_{t-1};\, \sqrt{\alpha_{t-1}}\rvx_0, (1 - \alpha_{t-1}) \mI
    \right) \enspace .
\end{align}
Though $q_\sigma(\rvx_{t-1} \mid \rvx_t, \rvx_0)$ depends on $\sigma$, $q_{\sigma}(\rvx_t \mid \rvx_0)$ actually does not.

The corresponding \textit{forward process} is again Gaussian though I do not need the Markov assumption as in DDPM
\begin{equation}
    q(\rvx_t \mid \rvx_{t-1}, \rvx_0)
    = \frac{q(\rvx_t, \rvx_{t-1}, \rvx_0)}{q(\rvx_{t-1} \mid \rvx_0) q(\rvx_0)}
    = \frac{q(\rvx_{t-1} \mid \rvx_{t}, \rvx_0) q(\rvx_{t} \mid \rvx_0) q(\rvx_0)}{q(\rvx_{t-1} \mid \rvx_0) q(\rvx_0)} \enspace .
\end{equation}

When $\sigma \to 0$ the reverse process is deterministic
\begin{equation}
    \lim_{\sigma \to 0} q_\sigma(\rvx_{t-1} \mid \rvx_t, \rvx_0) 
    = \dN \left( \rvx_{t-1};
    \sqrt{\alpha_{t-1}}\rvx_0
    + \sqrt{1 - \alpha_{t-1}} \, \frac{\rvx_t - \sqrt{\alpha_t}\rvx_0}{\sqrt{1-\alpha_t}}, \, 0 \mI
    \right) \enspace ,
\end{equation}
so that
\begin{equation}
    \rvx_{t-1} = \sqrt{\frac{1 - \alpha_{t-1}}{1-\alpha_t}}\rvx_t 
    - \sqrt{\frac{1 - \alpha_{t-1}}{1-\alpha_t}} \sqrt{\alpha_t}\rvx_0 
    + \sqrt{\alpha_{t-1}}\rvx_0 \enspace .
\end{equation}
In consequence the forward process is also deterministic
\begin{equation}
    q_{\sigma}(\rvx_t \mid \rvx_{t-1}, \rvx_0) = \sqrt{\alpha_t}\rvx_0 - \sqrt{\frac{1-\alpha_t}{1 - \alpha_{t-1}}} \sqrt{\alpha_{t-1}}\rvx_0 + \sqrt{\frac{1-\alpha_t}{1 - \alpha_{t-1}}} \rvx_{t-1} \enspace .
\end{equation}

As in DDPM we can sample $\rvx_t$ from the same distribution
\begin{equation}
    \rvx_t = \sqrt{\alpha_t} \, \rvx_0 + \sqrt{1-\alpha_t}\, \rvepsilon_t
\end{equation}
and similarly as before we can reverse this and predict the de-noised observation
\begin{equation}
    \widehat{\rvx}_0(\rvx_t) = \frac{\rvx_t - \sqrt{(1 - \alpha_t)} \, \epsilon^{(t)}_\theta(\rvx_t)}{\sqrt{\alpha_t}} \enspace .
\end{equation}
We also have 
\begin{equation}
    \epsilon^{(t)}_\theta(\rvx_t) = 
    \frac{\rvx_t - \sqrt{\alpha_t} \, \widehat{\rvx}_0(\rvx_t)}{\sqrt{(1 - \alpha_t)}} \enspace .
\end{equation}


Using this we can define the generative process starting from $\pt(\rvx_T) = N(0, I)$ and then
\begin{equation}
    \pt^{(t)}(\rvx_{t-1} | \rvx_t) = q_{\sigma}(\rvx_t \mid \rvx_{t-1}, \widehat{\rvx}_0(\rvx_t)) = 
    \dN \left( \rvx_{t-1};
    \sqrt{\alpha_{t-1}}\, \widehat{\rvx}_0(\rvx_t)
    + \sqrt{1 - \alpha_{t-1} - \sigma_t^2} \, \frac{\rvx_t - \sqrt{\alpha_t}\, \widehat{\rvx}_0(\rvx_t)}{\sqrt{1-\alpha_t}}, \, \sigma_t^2 \mI
    \right) 
\end{equation}
and
\begin{equation}
    \pt^{(1)}(\rvx_{0} | \rvx_1) = 
    \dN \left( \rvx_{0};\, \widehat{\rvx}_0(\rvx_1),\, \sigma_1^2 \mI
    \right) \enspace .
\end{equation}

The ELBO is again
\begin{align}
    \text{ELBO} & = \E_{q(\rvx_{0:T})} \left[ \log \pt(\rvx_T) + \log \prod_{t=2}^T 
    \frac{\pt^{(t)}(\rvx_{t-1} \mid \rvx_{t})}
    {q_{\sigma}(\rvx_{t-1} \mid \rvx_{t}, \rvx_0)} 
    + \log \frac{\pt^{(1)}(\rvx_{0} \mid \rvx_{1})}{q_{\sigma}(\rvx_{T} \mid \rvx_0)}
    \right] \nn
    & = \E_{q(\rvx_{0:1})} \log \pt^{(1)}(\rvx_{0} \mid \rvx_{1}) 
    - \E_{q(\rvx_{0})} \sum_{t=2}^T \KL\left( q_{\sigma}(\rvx_{t-1} \mid \rvx_{t}, \rvx_0) \mid\mid \pt^{(t)}(\rvx_{t-1} \mid \rvx_{t}) \right) \enspace .
\end{align}

We look at
\begin{align}
    & \KL\left( q_{\sigma}(\rvx_{t-1} \mid \rvx_{t}, \rvx_0) \mid\mid \pt^{(t)}(\rvx_{t-1} \mid \rvx_{t}) \right) \nn
    = & \KL\left( q_{\sigma}(\rvx_{t-1} \mid \rvx_{t}, \rvx_0) \mid\mid q_{\sigma}(\rvx_{t-1} \mid \rvx_{t}, \widehat{\rvx}_0(\rvx_t)) \right)  \nn
    = & \frac{1}{\sigma_t^2} \left\lVert 
    \sqrt{\alpha_{t-1}}\rvx_0
    + \sqrt{1 - \alpha_{t-1} - \sigma_t^2} \, \frac{\rvx_t - \sqrt{\alpha_t}\rvx_0}{\sqrt{1-\alpha_t}} -  \sqrt{\alpha_{t-1}}\widehat{\rvx}_0(\rvx_t))
    - \sqrt{1 - \alpha_{t-1} - \sigma_t^2} \, \frac{\rvx_t - \sqrt{\alpha_t}\widehat{\rvx}_0(\rvx_t))}{\sqrt{1-\alpha_t}}
    \right\rVert_2^2 \nn
    = & \frac{1}{\sigma_t^2} \left\lVert 
    \sqrt{\alpha_{t-1}} (\rvx_0 - \widehat{\rvx}_0(\rvx_t)) - \frac{\sqrt{1 - \alpha_{t-1} - \sigma_t^2}\sqrt{\alpha_t}}{\sqrt{1-\alpha_t}} (\rvx_0 - \widehat{\rvx}_0(\rvx_t))
    \right\rVert_2^2    \nn
    = & \frac{1}{\sigma_t^2} \left\lVert 
    \frac{\sqrt{1-\alpha_t}\sqrt{\alpha_{t-1}} - \sqrt{1 - \alpha_{t-1} - \sigma_t^2}\sqrt{\alpha_t}}{\sqrt{1-\alpha_t}} (\rvx_0 - \widehat{\rvx}_0(\rvx_t))    
    \right\rVert_2^2 \nn
    = & \frac{(\sqrt{1-\alpha_t}\sqrt{\alpha_{t-1}} - \sqrt{1 - \alpha_{t-1} - \sigma_t^2}\sqrt{\alpha_t})^2}{\sigma_t^2 (1-\alpha_t)} \left\lVert 
    \rvx_0 - \widehat{\rvx}_0(\rvx_t)    
    \right\rVert_2^2 \nn
    = & \frac{(\sqrt{1-\alpha_t}\sqrt{\alpha_{t-1}} - \sqrt{1 - \alpha_{t-1} - \sigma_t^2}\sqrt{\alpha_t})^2}{\sigma_t^2 (1-\alpha_t)} \left\lVert 
    \frac{\rvx_t - \sqrt{(1 - \alpha_t)} \, \epsilon_t}{\sqrt{\alpha_t}} - \frac{\rvx_t - \sqrt{(1 - \alpha_t)} \, \epsilon^{(t)}_\theta(\rvx_t)}{\sqrt{\alpha_t}}    
    \right\rVert_2^2 \nn
    = & \frac{(\sqrt{1-\alpha_t}\sqrt{\alpha_{t-1}} - \sqrt{1 - \alpha_{t-1} - \sigma_t^2}\sqrt{\alpha_t})^2}{\sigma_t^2 \alpha_t} \left\lVert 
    \epsilon_t - \epsilon^{(t)}_\theta(\rvx_t)
    \right\rVert_2^2 
\end{align}

This is up to the terms before the norm which do not depend on $\theta$ the same as in DDPM.
Since we wish to maximize ELBO which is a sum of these KLs across $t$, the optimal solution is reached when each of the norms is minimized irrespective of the weighting. 


\section{Sampling from DDIM}\label{sec:ddim_sampling}

We had before that 
\begin{equation}
    \pt^{(t)}(\rvx_{t-1} | \rvx_t) = q_{\sigma}(\rvx_t \mid \rvx_{t-1}, \widehat{\rvx}_0(\rvx_t)) = 
    \dN \left( \rvx_{t-1};
    \sqrt{\alpha_{t-1}}\, \widehat{\rvx}_0(\rvx_t)
    + \sqrt{1 - \alpha_{t-1} - \sigma_t^2} \, \frac{\rvx_t - \sqrt{\alpha_t}\, \widehat{\rvx}_0(\rvx_t)}{\sqrt{1-\alpha_t}}, \, \sigma_t^2 \mI
    \right) 
\end{equation}
and hence we can sample
\begin{align}
    \rvx_{t-1} & = \sqrt{\alpha_{t-1}}\, \widehat{\rvx}_0(\rvx_t)
    + \sqrt{1 - \alpha_{t-1} - \sigma_t^2} \, \frac{\rvx_t - \sqrt{\alpha_t}\, \widehat{\rvx}_0(\rvx_t)}{\sqrt{1-\alpha_t}} + \sigma_t \epsilon_t \nn
    & = \sqrt{\alpha_{t-1}}\, \left( \frac{\rvx_t - \sqrt{(1 - \alpha_t)} \, \epsilon^{(t)}_\theta(\rvx_t)}{\sqrt{\alpha_t}} \right)
    + \sqrt{1 - \alpha_{t-1} - \sigma_t^2} \, \epsilon^{(t)}_\theta(\rvx_t) + \sigma_t \epsilon_t \nn
\end{align}






\clearpage 
\printbibliography



% print index
\phantomsection
\cleardoublepage
\addcontentsline{toc}{section}{\indexname}
\printindex

\end{document}